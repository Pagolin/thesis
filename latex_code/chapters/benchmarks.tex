\section{Test Setup}
\section{The YCSB}

The "Yahoo! Cloud Serving Benchmark" YCSB \cite{YSBC} comprises a framework to compare the performance of different database and key-value-store implementations. The two key components are an adaptable client that generates workloads for a connected database and a set of core workload scenarios resembling common use cases of data base deployments. The open source implementation of YCSB\footnote{\url{https://github.com/brianfrankcooper/YCSB/wiki}} already includes the benchmarking setups for many commonly used systems as DynamoDB, Redis, Cassandra, MongoDB and many more. Adding new test targets is possible via an abstraction layer for the concrete database API. This abstraction layer called \code{database interface layer} has to provide concrete implementations for common database commands i.e. read, insert, update, delete and scan for the new data base. Also the YCSB client needs to be set up to connect to the concrete data base. \\

The problem we face when we try to use as a benchmark suite is, to actually connect it with an application running on M$^3$. The reason is, that we can not simply run YCSB itself on M$^3$, because it is written in Java and M$^3$ doe not provide a Java runtime. 
The solution, the authors themselves used to benchmark the integrated database on M$^3$ was to use YCSB only to generate requests. Those requests where written to a file and custom importer and client apps where implemented to load the requests and send them to the M$^3$ levelDB implementation in benchmark tests. 

YCSB allows to specify the test scenario as so called \emph{workloads}. Those workloads define in particular the following properties, defaulting to the following values 
Workloads:
\begin{itemize}
    \item number of records to generate and insert in the loading phase, default 1000
    \item number of operations (CRUD) to perform on the server, default 1000
    \item relative share of the individual operations, e.g. 50\% Read, 10\% Write a.s.o.
    \item number of client threads, default 1
    \item type of measurement 
\end{itemize}

There is also more involved options like access to hot spot data, distribution of data access or access or insertion using sorted or unsorted keys. However the target of our measurement is not the actual data backend but the network processing of requests and responses. So we focus on frequency and size of packets to compare the two network stack implementations.

The workloads to runn are defined in separate text files. 
