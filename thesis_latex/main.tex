\documentclass[diplominf]{zihpub}

\usepackage{color}
\usepackage{float}
\usepackage{hyperref}
\usepackage{minted}
\usepackage{multirow}
\usepackage{pdfpages}
\usepackage{pbox}
\usepackage{subcaption}
\usepackage{todonotes}


% Listings mit subfigures
\usepackage{newfloat}
\usepackage{subcaption}
\DeclareFloatingEnvironment[name={Listing}]{codefigure}

% Just for convenience

\newcommand{\question}[1]{\todo[inline, color=blue!40]{#1}}
\newcommand{\note}[1]{\todo[inline, color=yellow!40]{#1}}
\definecolor{rubinered}{HTML}{ED017D}
\newcommand{\seflue}[1]{\todo[inline, color=rubinered!40]{#1}}

\setminted{fontsize=\footnotesize}
\setmintedinline{fontsize=\small}
\newcommand{\means}{$\rightarrow{}$}
\newcommand{\code}[1]{\textcolor{gray}{$\mathsf{#1}$}}
\newcommand{\rust}[1]{\mintinline{rust}{#1}}
\newcommand{\dev}[1]{\rust{device#1}}
\newcommand{\stack}[1]{\rust{ip_stack#1}}
\newcommand{\store}[1]{\rust{store#1}}
\newcommand{\md}[0]{M$^3\ $}


\author{Lisza Zeidler}
\title{Compiling Unikernels into Micro Kernels}
\matno{4545705}
\betreuer{}

%\bibfiles{literature.bib}

\abstracten{
With increasing complexity, programs become more prone to bugs and security vulnerabilities. This is particularly true of kernels. For example, the original feature set of the monolithic Unix kernel is still continuously extended by functions, drivers and modules. Since these are not mutually constrained, each additional component increases the attack surface of the entire system. There are several approaches to solving this problem and implementing the concept of defense in depth. However, they all involve runtime costs and, most importantly, manual customization. This effort makes it difficult or impossible to flexibly adapt existing software to isolation mechanisms that provide an appropriate trade-off between security and performance overhead.\\ 

The idea of this work is to investigate whether the isolation of components of a server application can also be done by a compiler. The Ohua compiler has been developed to identify independent execution steps in a sequential program and to transform the program into a dataflow program consisting of independent nodes with potentially separate memory. The specific implementation of the nodes is determined by architectural integrations. Nodes can be threads or processes, or theoretically other isolation concepts.  We wanted to use Ohua to convert a server application, in which the application, IP stack and network interface share the same memory, into an application for the microkernel-based operating system \md. The main questions were i) how to restructure the server application so that in the resulting dataflow graph the application, IP stack and network interface each operate in exactly one isolated node, and ii) could these restructurings be implemented as compiler transformations. \\

We show how the application can be restructured accordingly. Formal descriptions already exist for some of the transformations. However, it has also become clear that the syntax of the input program alone does not contain enough information to define, for example, whether or not the program should be adapted to concrete target systems such as \md. Therefore, this thesis discusses which transformations could be implemented as compiler transformations in the future, and which transformations still have to be done by the programmer. }


\acknowledgments{
Of course, I would like to thank the people who helped me with this thesis. To my supervisor Sebastian Ertel for his time, help and brainstorming together, as well as Felix Suchert for helping me master Rust problems and Nils Asmussen for helping me understand \md. \\ 

But I wouldn't have written this thesis, and probably not a single line of code, without Vera Zeidler and Sebastian Fl\"ugge. Without Sebastian I would not have had the idea to study computer science, and that would have been a pity, because the idea was really good. Without Vera I literally wouldn't be here. Thank you for all your efforts, love and cleverness, I hope you are also a little proud.}

\begin{document}
\begin{thesisdocument}

\chapter{Introduction}
\label{Chapter:Intro}
Whenever tasks become more extensive and complex, systems react with division of work and specialization. By subdividing into smaller, less complex tasks, large processes can be distributed and processed in parallel, and at the same time the executing components are more efficient because they have to cover less noise, peripheral cases and side aspects. We can observe this development on different scales in computer science. At the hardware level, computations are distributed from general-purpose CPUs to GPUs, digital signal processors (DSPs) or other adapted hardware, and the concrete hardware logic is abstracted from the program logic to be executed. At the software level, components are separated both vertically, i.e. between business logic, language runtime environment and operating system services, and horizontally, i.e. between components of the different levels of independence. \\

Besides improved efficiency and scalability, compartmentalization has also a major advantage when it comes to security. Separation of concerns and minimization of trust assumptions are core concepts in defense in depth for cloud deployments. \seflue{"core concepts in defense in depth for cloud deployments" ... K\"ommer das nochmal umschreiben, ich raffs nicht.} But again they are likewise seen on the scale of single machines. A cloud deployment consisting of one big trust zone protected by a perimeter is the large scale equivalent of a monolithic kernel, with unrestricted memory access among all kernel space processes. The problem with this is clearly evident in monolithic Unix kernels. Despite various measures to prevent unauthorized memory access (e.g. control flow integrity and data execution prevention) or to make it more difficult to exploit security vulnerabilities (e.g. address space layout randomization and stack canaries), according to statistics of the NIST\cite{nvd} the number of vulnerabilities in the kernel ecosystem is still increasing. An analysis of the NIST National Vulnerability Database in \cite{mckee2022novel} looked at the reasons for critical vulnerabilities over the last 5 years and concluded that 34\% of them were due to lack of storage security, and another 43\% were due to lack of compartimentalization. The authors also pointed out that these vulnerabilities could have been prevented if essentially independent processes could not access shared memory, but that however the so-called least-privilege policy is not enforced by the Linux kernel or other big open source projects as OpenSSL or the Apache Server. 
Similarly, as the authors report in \cite{kirth2022pkru}, the majority of known vulnerabilities in Windows, Chrome and the Android Open Source Project (AOSP) can also be traced back to insecure memory access. \\

So obviously it is desirable to have compartmentalization and data locality enforcement not only in cloud setting but also in the kernel itself. Existing solutions for memory protection are tied to specific hardware requirements as Memory Protection Keys, Trusted Execution Environments like ARMs Trust Zone\cite{pinto2019demystifying} or Intels SGX \cite{costan2016intel} or specific virtualization layers (\cite{tan2007ikernel}, \cite{nikolaev2013virtuos}) and in general require manual code adaptations. The costs of adapting tends to make migration between architectures and to newer solutions more difficult and leads to components being subdivided more coarsely than would make sense in order to save effort and runtime costs. Also in terms of security, compartmentalization and in particular concurrency comes at a cost. Recent trends in secure computing are strongly moving away from simple testing towards verification and proving. This applies to the proof of certain properties of user programs as well as to the verification of compilers, kernels ,or operating systems (e.g. CompCert\cite{leroy2009formal},\cite{sL4Verf}, \cite{gu2016certikos}). Distributed, concurrent programs, however, are much more difficult to verify. \\

While certain costs are unavoidable when program parts are isolated, the costs of manually adapting to different isolation mechanisms are not among them. The classic approach to separating concrete run-time implementations from program logic is compilers. They allow the programmer to define and test the logic within specific programming models and automatically translate and add to them to fit the chosen runtime environment and architecture. 

This work will therefore also be based on Ohua\cite{ertel2015ohua}, a compiler developed to identify independent processing steps from sequential programs and deploy them to concurrent, potentially isolated nodes of a data flow graph. Ohua works on subsets of high-level languages such as Python or Rust. Instead of machine code, Ohua extracts a dataflow program from the input, introducing two main abstractions, the notion of an independent node and the notion of communication edges, and replaces them with the corresponding implementations for different run times. The idea of this work is to use and extend Ohuas capabilities for program transformation. We want to be able to extract isolated components from a shared memory program suitable for unikernels and automatically derive the code to deploy them in a microkernel setting. Concretely we will use the \md operating system as an example backend to provide process isolation. 

The key question we need to answer is: What transformations are needed to turn a monolithic or unikernel program into one in which isolated states work together in a data flow graph? 

A good example case to approach the answer to this question is are server applications. They are quasi-ubiquitous in distributed applications and consist of various components, in particular the data backend, the TCP/IP-Stack and a network driver that should be isolated from each other for security reasons. To better understand the starting point and requirements of this work, in Chapter~\ref{Chapter:Background} we first consider the function of Ohua and the properties of the target architecture \md. Certain properties of Rust's type system are explained in this chapter, as they form the basis of Rust's storage security and are helpful constraints for meaningful transformations. On this basis, Section~\ref{Chapter:Implementation} first describes the example application. In a rough sketch, we first approach the structural problems and describe schematically how the application should function after the transformations. We then describe how individual aspects of the code must be transformed.
One of the main problems we need to solve is how to achieve state locality. Obviously the components of a 

\begin{itemize}
    \item develop and verify the sequential, local code \means have verified transformations applied \means get a verified distributed program with isolation boundaries inserted automatically
\end{itemize}

Actual Problem description

    \begin{enumerate}
        \item Transformation to message passing: In it's current state, smoltcp actually combines the TCP/IP stack and an interface to the actual physical layer. The according parts of the library are coupled via shared references and invoke each other via normal function calls. This works well in unikernels, as well as in monoliths (implemented in user space vs. kernel space respectively). In a microkernel, we might want to separate this two functionalities to use and scale them independently. This requires components to interact not via function calls and shared references but via IPC. So one aspect of the problem is to investigate which constraints the isolation of components imposes on the code and which transformations achieve a compatible structure. 
        \item Local State: Smoltcp is a good example for an application with interacting states. From a bird's eye view we can basically distinguish three states, the server application holding e.g. requests currently to process, the network stack holding for example socket states and the network interface abstraction
    
        \item the state locality problem a) write the code such that no internal function will try to access another components state outside the 'flow' b) rewrite it such a way that Ohua identifies the components we want
        
        \item Compile from Uni- to Microkernel \question{$M^3$ already is a Microkernel using smoltcp as a userspace service, so how to describe exactly what's the point here?}
    \end{enumerate}

    \begin{figure}[H]
    \centering
    
    \begin{subfigure}[b]{0.6\textwidth}
         \centering
         \begin{minted}{rust}
   fn somefun() -> i32 {
       let mut mStruct = MS::default();
       let i = mStruct.f();
       let j = mStruct.g();
       calculate(i,j)
   }       
            \end{minted}
         \caption{Multiple calls to a struct}
         \label{multCalls}
     \end{subfigure}
     %\hfill
     \par\bigskip
     \begin{subfigure}[b]{0.6\textwidth}
         \centering
         \includegraphics[width=\textwidth]{figures/graph_state_moved.png}
         \caption{Resulting data flow graph}
         \label{DFGStateMoved}
     \end{subfigure}
    \par\bigskip
     \begin{subfigure}[b]{0.6\textwidth}
         \centering
         \includegraphics[width=0.6\textwidth]{figures/graph_state_not_moved.png}
         \caption{Target data flow graph}
         \label{DFGStateNotMoved}
     \end{subfigure}
    \caption{}
    \label{fig:SimpleStateUseFigure}
    \end{figure}
    
    \todo[inline]{Explain the contingency from SharedMemory programming model, over M3 to Cloud deployment: maybe Mention in Intro, details in the background}
    Describe essential differences at: 
    \begin{itemize}
        \item Is it one program or do we compile/deploy different programs?  $\rightarrow$ are types known/inferable at compile time/do they have to be? $\rightarrow$ Do we compile for the same OS? \\
         $\Rightarrow$ points above have consequences e.g. 
        \item What are the constraints on types being passed (serializable, are \rust{dyn} allowed, are generics allowed, who takes care of compatibility in case of different OSs)? 
        \item What are other constraints for the programmer (e.g. using system specific interfaces for components) How much packaging does the Compiler have to do (e.g. assorting libraries to components)? 
        \item (How) do we need to handle errors/panics?(in a 'one program' scenario it's one panic to crash them all. How is it in M3? It's even more complex in 'really' distributed scenarios as there are different reasons components might not answer and even if they answer with an error we might have different strategies to handle.)
        \item Do we need to build in a shut down mechanism? 
        \item Can/Should the Compiler be able to detect system interaction and tailor them?        
    \end{itemize}
    
\todo[inline]{Important Point I don't know where to put yet and also needs elaboration:}
\note{The transformation we will extract from the rewriting process are to some degree actually extensions of Ohua programming model. The model currently already enforces 1) variables to be either used as state, or as variable i.e.\means no sending of states and 2) states from outside the loop scope to be only used once inside loop or recursion \means linearity inside loops. What is not enforced currently is that states in general are used only once, i.e. outside loops or when created inside loops states can be used more than once \means so no linearity here.}

    

From the task definition
\begin{itemize}
    \item Implement the cloud-unikernel using SmolTCP, a well-established networking library -> Also schlicht eine Beispielanwendung mit smoltcp, die Anfragen an einen Key-Value-Store handelt. 
    \item Rewrite the unikernel, such that Ohua can compile it Derive and implement transformations to make state usage local to a single program location to provide isolation.
    \item Update existing M3 Backend
    \item Evaluate the approach in the (existing) setup of the YCSB key-value store benchmark along the performance-safety trade-off. 
\end{itemize}

\todo[inline]{Example Rust 4 Unikernel \cite{Rust4Unikernel}}
\todo[inline]{Formal Verification eg for L4 derivatives as motivation \cite{sL4Verf}}

\todo[inline]{Can I get e reference to Compositionality papers .. It would be awesome if we could get the link to \means "we transform a verified object into a verified category, i.e. composable, verified objects"}

\todo[inline]{Certified Concurrent OS in \cite{gu2016certikos} "We have successfully developed a practical concurrent OS kernel and verified its (contextual) functional correctness in Coq. Our certified kernel is written in 6500 lines of C and x86 assembly and runs on stock x86 multicore machines. To our knowledge, this is the first proof of functional correctness of a complete, general-purpose concurrent OS kernel with fine-grained locking."}




\chapter{Background}
\label{Chapter:Background}
The goal of this work is to understand which transformations are necessary to convert a program from a sequential programming model with shared memory into a distributed, concurrent programming model of a microkernel operating system. A significant part of these transformations is already implemented in the Ohua compiler. Therefore, the specific goal is to understand and close the 'translation gaps' between the programming models of the source program, the current Ohua implementation, and the M3 operating system. So to better understand the work that needs to be done, we will introduce those models in this chapter.

\section{Ohua}
\label{sec:back_ohua}
In this section we will introduce the Ohua compiler\cite{ertel2015ohua}. Its essential concept is to extract the underlying data flow graph from a sequential input program. The result is a program structure consisting of individual steps that are connected only by incoming and outgoing data and can be executed concurrently. These individual steps and their data channels can then be mapped to various abstractions of processes and communication channels in the backend of the compiler to achieve parallelization and isolation of the steps. Like most compilers, Ohua works step-by-step with different intermediate representations (IRs) of the input program. To be able to support different languages and target architectures, Ohua uses language integrations. Currently there are integrations for Rust and Python. In the next subsection, we will take a closer look at the basic structure and function of Ohua. 

For this work, it is also important to understand what programming model Ohua currently supports. Here, programming model means on the one hand which restrictions in the syntax of the input programs are currently necessary to be able to convert them into correct concurrent programs. On the other hand it contains the explicit and implicit assumptions about the concrete implementations of 'processes' and 'channels'. So we will also look at these aspects in more detail in this section.

\subsection{Compiler Pipeline}
\label{subec:OhuaPipeline}
When we say 'Ohua compiles a program', we mean it compiles functions in the compile scope. In contrast to e.g. rustc or gcc, Ohua does not compile the complete code of the program, but transforms only the functions, for example within one or more Python modules specified in the call. We call these functions algorithms. In contrast to algorithms, functions and methods that are imported and used within algorithms are not compiled. They are completely opaque to the compiler. This also means that Ohua does not require any syntax constraint in these imported functions and methods. An overview of the compiler pipeline is shown in Figure~\ref{fig:ohua_fine}.  

In the \textbf{compiler frontend}, algorithms of the target language are first parsed and translated into Ohua's frontend language (Frontend IR). This translation is implemented in language specific frontend integrations. That is, for each language supported by Ohua, such a frontend integration must exist. This integration parses algorithms of the input language and translates them into the syntax of the frontend IR. For non-supported syntax constructs, currently for example \rust{break} statements in loops, the compilation terminates at this point. The currently supported subset of Rust can be found in the Appendix~\ref{sec:RustIntegration}.

\begin{figure}[H]
    \centering
    \includegraphics[scale= 0.36]{figures/ohua_fine_with_channels.png}
    \caption{Structural overview of Ohua and the code transformations}
    \label{fig:ohua_fine}
\end{figure}

In the \textbf{core compiler} itself there are two main representations of the code. One is Expression IR. This language is functional and based on the call-by-need lambda calculus. To transform input algorithms to this representation 
calls to other algorithms are inlined, renaming compiler passes ensure single static assignment form, and assignment expressions are refactored to applicative normal form. A simplified example of code before and after the transformation is shown in Figure~\ref{fig:funBodyTranslation}. A central conversion step in the compiler is the transformation of stateful calls to so called \emph{state threads}(\cite{wadler1992essence}, \cite{launchbury1994lazy},\cite{ertel2019stclang}). To generate race condition free tasks, Ohua forbids shared state use and only permits stateful computation inside methods. However, methods in an imperative language mutate objects in place and implicitly refer to the new, changed state by the same reference as before. The conversion of stateful calls to state threads makes the semantic of creating a new state upon calling methods explicit. In the code example in Fig.\ref{fig:funBodyTranslation}, the call \rust{someState.do()}, is internally transformed to explicitly take a state as an argument and return a new, mutated state. Thereby, function calls downstream to not need to access a shared memory to use stateful objects. Based on this explicit state threading, Ertel et al. \cite{ertel2019stclang, ertel2018supporting} developed functional representations for imperative control flow on stateful computations. For example, an imperative for-loop is translated to a so called \rust{smap} operation, which is essentially a fold operation of the loop body on the states manipulated inside the loop. \\

\begin{figure}
    \centering
    \begin{subfigure}[b]{0.45\textwidth}
         \centering
         \begin{minted}[fontsize=\small]{rust}
 fn algo(i){
   let someState = 
           other_algo(i);
   let a = someState.do()
   let b = f(a)
   return b
 }

 fn other_algo(i){
    let s = State::new(i)
    s
 }
            \end{minted}
         \caption{Input algorithm}
         \label{simplPyInput}
     \end{subfigure}
     \hfill
     \begin{subfigure}[b]{0.5\textwidth}
         \centering
         \begin{minted}[fontsize=\small, escapeinside=||,mathescape=true]{haskell}
let someState = 
        |$\lambda$| State::new (i) in
  let a, someState_0 = 
           |$\lambda$| do (someState, a) in
    let b = |$\lambda$| f (a) in
        b
        \end{minted}
        \vspace{15mm}
    \caption{Pseudocode of IR}
         \label{simplIR}
    \end{subfigure}
\caption{An algorithm is mapped to a nested let-expression with the innermost term representing its return value}
\label{fig:funBodyTranslation}
\end{figure}


The next representation in the compile flow is the Data Flow Graph (DFG) representation. Independent program tasks are encapsulated in the this representation and are explicitly assigned their incoming outgoing data channels. Besides function calls from the original program, this representation also contains control nodes that govern the data flow. For example if the input code contained a branching statement like \rust{if cond {f()} else {g()}} control nodes will be introduced to a) switch data flow between calls to \rust{f()} or \rust{g()} and b) to collect results from appropriate output channels of \rust{f()} or \rust{g()} depending on the condition. This representation also allows to merge certain nodes by fusing their code, as well as input and output channels. This is done for instance if a following node entirely depends on its predecessor and has so little work to do that it would hardly justify the overhead of spinning up an independent task in any backend implementation.\\

Which ultimately brings us to the \textbf{compiler backend} and backend integrations of Ohua. Backend integrations consist of two parts. The 'Language Backend' is only language specific. Similar to the frontend integration, it serves the purpose of translating code inside tasks from Ohua representation syntax back to the target language syntax. The 'Architecture Backend' is responsible for translating the 'nodes' and 'edges' of the DFG into a specific implementation for concurrent tasks, communication channels and a runtime for the graph. For example, there can be two different architectures for a given language: one implementing tasks as threads and one using processes, with both also generating appropriate channels and runtime code to execute the DFG. As we did not compile imported functions, the target language must match the input language, which is automatically ensured by the compiler. Architectures for the same language can be used interchangeably. This way Ohua can generate e.g. multi-threaded shared memory or fully distributed programs from the same input.\\

In the next section we will take a closer look at the restrictions and assumptions required to make this work.

\subsection{Programming Model}

The term programming model generally describes a relationship between syntax constructs in a programming language and their concrete semantics in a particular execution environment. In the case of Ohua, the programming model includes, on the one hand, the supported input syntax and the assumptions made about the supported terms of the input language. On the other hand, it specifies how these terms are translated into a dataflow graph and what assumptions Ohua makes about the implementation of nodes, edges, and runtime of the DFG. First we will look into the supported input syntax and assumed semantics. 

\subsubsection{Input Syntax and Semantics}
We already know that Ohua's basic input units are algorithms, i.e. pure functions inside the compile scope. Figure~\ref{tab:FrontendIR} depicts the language definition of the frontend representation described before. Any syntax construct of the input language has to be mapped to the according terms of this language to be compiled. In the following paragraphs we will describe the accepted syntax constructs and the semantics the programming model expects them to have. \\


\begin{table}[ht]
\resizebox{\columnwidth}{!}{%
    \begin{tabular}{l c l l}
        \multicolumn{4}{l}{\emph{Patterns:}}\\
        $p$ & $::=$ & $x\ |\ (x,~\ldots ,~x)\ |\ ()$ & named variables, tuples or unit\\
        \multicolumn{4}{l}{\emph{Expressions:}}\\
        $e$ & $::=$ & $e$ & named expression in host language\\
        & $|$ & $\textbf{1},\textbf{2},\textbf{3}, \ldots \ |\ \textbf{true}\ |\ \textbf{false}\ |\ \textbf{()} $ & typed literal in host language\\
        & $|$ & $\textbf{let}\ p\ \textbf{=}\ e\ \textbf{in}\  e$ & lexical scoping \\
        & $|$ & $e\ e$ & application\\
        & $|$ & $\boldsymbol{\lambda} [p,~\ldots ,~p]\textbf{.}\  e$ & abstraction \\
        & $|$ & $\textbf{if}\ e\ \textbf{then}\ e\ \textbf{else}\ e$ & conditionals\\
        & $|$ & $\textbf{map}\ e\ e$ & map first expression to second\\
        & $|$ & $\textbf{bind}\ e\ e$ & bind an expression representing a state to  \\
        & & & an expression representing a function to act on this state \\
        & $|$ & $\textbf{stmt}\ e\ e$ & expression whose return value is ignored\\
        & $|$ & $\textbf{seq}\ e\ e$ & \\
        & $|$ & $\textbf{(}\ e\ \textbf{)}$ & tuple of expressions \\
    \end{tabular}%
    }
    \caption{Definition of the Expression IR}
    \label{tab:FrontendIR}
\end{table}

\textbf{Function Calls:} Beside algorithms, Ohua supports stateful and stateless function calls, i.e. methods and pure functions, imported into the scope. Pure functions are expected to be side effect free. In particular, the programming model assumes that pure functions do not implicitly manipulate their arguments. This excludes, for instance, functions that manipulate their arguments by reference. If the output of a pure function call is not used, it is considered to have no effect and is removed during compilation. 

Stateful function calls, on the other hand, are expected to manipulate the object they are called on, i.e. have a side effect. Consequently, they are not removed regardless if the output is used. 
Any stateful computation is expected to happen exclusively and explicitly using method calls and also method calls are expected to only manipulate the object state itself. This also entails the requirement that the state is not 'leaked' via return values. For example, in the method call \rust{let x = SomeState.do_stuff();}, \rust{x} must not be, or contain, a reference to \rust{SomeState}. We already assumed that other functions do not manipulate \rust{SomeState} when using \rust{x} as argument. However, without this 'leaking assumption' it would be possible to call \rust{x} as a stateful object, thereby implicitly manipulating the state of \rust{SomeState}. This implicit semantic is not handled currently and would be lost in the distributed output code. 

Functions need to be typed or type-able by the frontend integration. To correctly annotate the types in generated code, at least to the extent required for any particular backend and architecture, Ohua needs to extract type information from the input code. Specifically the argument types of each function call are extracted and preserved in the different IRs as typed function literals capturing the argument types of each function call. \\

\textbf{Loops}: Ohua supports bound and unbound \rust{for}-loops. They are transformed into parallelizable pipelining of the independent calculation steps inside the loop. Inside for-loops, each state from outside the loop must be used at most once to enable the accumulation of state changes in a single node for each object. Conditional loops (\rust{while} or \rust{do-while}) are as of the beginning of this work not supported, but can be expressed using recursion.\\

\textbf{Recursion}: Ohua supports recursion with some notable restrictions. Recursive algorithms must be tail recursive, the recursive call must be located in the if-branch, and the return value must be in the else-branch of recursion. Furthermore, either one must be the only statement in each branch respectively. As return values only single variables are supported. Recursive loops will not yield any pipelining or parallelization but a loop executed for one input at a time. This is due to the semantic of recursion, being a repeated function on a state, where the result of each step depends not only on that step but also on previous results.
Currently, the output of recursive algorithms cannot be used in an assignment, i.e. a recursive call can only be the last statement in another algorithm and return the calling algorithm's final result. \\

\textbf{Branching:} Branching is supported in case of simple if-else expressions, where both branches must be present. Also, if-else statements, as for instance present in the Python syntax, are currently not supported. This is because those statements have a different execution semantic than expressions, i.e. branches that do not return a value but have side effects on variables from the surrounding scope, so they need to be implemented separately. In addition, it is currently not possible to use stateful functions in branches.\\

\textbf{Return, Continue, Break:} Ohua currently does not support any forms of early return of conditional execution except for recursion and branching as described before. Therefore, \rust{break} and \rust{continue} are generally not supported at the moment, while \rust{return} is supported only for Python and only at as the last statement of a function block, because contrary to Rust there is no implicit value return in Python. \\

\textbf{Variables and Literals:} There are two categories of variables. Local variables are bound inside algorithms, environment variables are bound in outer scope. Thus, environment variables are basically arguments of the algorithms, but can also be imported or globally defined names. Ohua supports mutable and immutable local variable bindings. Local variables can either be used as a state, or as an argument to a function call. If it is used as an argument it can only be used once, if it is used as a state, it may be used more than once except, as explained before, inside loops. Environment variables cannot be used as a state directly\footnote{This changed during the course of this work. Now arguments to algorithms can be used directly as a generator for a for-loop.}, but can be used several times as function call argument. The underlying assumption of this distinction is that environment variables will be available in scope for all nodes created from an algorithm, while locally bound variables are sent to the consuming node. This assumption becomes relevant in architectures where the generated tasks have no access to a common global scope. In those cases, environment variables are not available to the task via a closure mechanism. This will be the case for M3 tasks. Finally there is a limited set of literals that is directly supported in the input. This includes integers, booleans, strings and unit literals. Other literals must be wrapped in a function call currently (simple binary operations are sufficient here e.g. to compile \rust{let x = 17} one could write \rust{let x = 17 + 0}).\\

Many of the current limitations are only due to lack of implementation. For example, there is no formal reason to allow recursive calls only in the if branch. These limitations will be fixed in the future. At the moment, however, they are the main reason why control flow expressions cannot be freely combined. That is, there are currently restrictions on the frontend language that are not reflected in the language itself.


\subsubsection{Enforcement}

Conformity with the allowed syntax subset is automatically implemented by each language integration, as it has to translate the input syntax to the frontend language. However, this is only a syntactical conversion. Except for tracking the annotated type of named variables, the compiler does no further semantic analysis of the input code. This means that to comply with a programming model requirement it is sufficient to match the expected syntax, but not necessarily the expected semantics. Take for example the requirement to use each variable only once as a function input. To use a variable \rust{x} twice, a programmer needs to return it twice from a function call \rust{let (x1, x2) = fun()}. However, she can freely decide whether \rust{x1} and \rust{x2} are copies or only references of \rust{x}, matching the expected syntax but not the expected semantics of the programming model. The result might still be valid output, but it is the responsibility of the programmer to ensure validity of reference passing in the concurrent output. Likewise, the use of global mutable state can be encapsulated in function calls. \\

In general, enforcement of the programming model is currently not separately implemented and in some case lacking completely. Violations of the programming model will either lead to runtime errors during any point of compilation or may lead to invalid output code. The latter case was not an obvious problem in Rust, as Rust itself enforces borrowing rules and therefore a considerable part of Ohua's limitations. However, this is not generally the case in other languages, so tests were added in the course of this work to ensure compilation failure upon violations of the programming model.


\subsubsection{Backend Language and Process Abstractions}
\label{subsec:BackendRequirements}
The terms of Ohua's backend language are depicted in Table~\ref{tab:DFGdef}. As complex syntax constructs of the input language are removed in the frontend, most of the terms in the backend language are basic constructs of any imperative language, as variables, literals, assignments, and simple control flow terms. To generate the code for Ohua-introduced control nodes, some \emph{specific functions} are also required to be present in the host language. In particular, the control of loop execution requires implementations for list handling, as well as the functionality for \emph{iterable objects} of the host language, to test if they have a fixed size and if so retrieve this size information. \\

Obviously, the backend language also entails the notion of named and typed channels, their sending and receiving ends, and sending and receiving of variables a means of communication among the tasks of the data flow graph. 

\begin{table}[ht]
\resizebox{\columnwidth}{!}{%
    \begin{tabular}{l c l l}
        \multicolumn{4}{l}{\emph{Typed Task Expressions:}}\\
        $e$ & $::=$ & $ x\ | \textbf{1}, \ldots \ |\ \textbf{true}\ |\ () $ & variables, simple literals \\
         & $|$& $ \textbf{funRef}\ | ref \textbf{envRef}\ !HostExpr  $ & function and environment references \\
        & $|$ & $ x (e,~\ldots,~e) |\ Obj.x (e,~\ldots,~e)  $ & pure function and method calls\\
        & $|$ & $\textbf{let} x\ \textbf{=}\ e\ \textbf{in}\ e |\ \textbf{=}\ e$ & scoped bindings and assignments \\
        & $|$ & $\textbf{stmt}\ e\ e$ & statement \\
        & $|$ & $x\textbf{\_receiver.receive()} $ & expression to  receive data \\
        & $|$ & $x\textbf{\_sender.send(}x\textbf{)} $ & expression to send data \\
        &\multicolumn{3}{l}{--control flow--}\\
        & $|$ & $\textbf{while True:}\ e$ & \\
        & $|$ & $\textbf{for}\ x\ \textbf{in}\ x\ \textbf{do}\ e$ & \\
        & $|$ & $\textbf{repeat}\ (x\ |\ l)\ e$ & \\
        & $|$ & $\textbf{while}\ e\ e$ & \\
        & $|$ & $\textbf{if}\ e\ \textbf{then}\ e\ \textbf{else}\ e$ & \\
        &\multicolumn{3}{l}{-- specific functions, required for control nodes --}\\
        & $|$ & $\textbf{newList}\ $ & create a list \\
        & $|$ & $\textbf{append}\ x\ e $ & append t to x \\
        & $|$ & $\textbf{hasSize}\ x$ & $[a]$ \means Bool \\
        & $|$ & $\textbf{size}\ x$ & $[a]$ \means Int \\
        & $|$ & $\textbf{(}\ l,\ l\textbf{)}\ | \  \textbf{(}\ x,\ x\textbf{)}$ & Tuple of literals or bindings\\
        & $|$ & $(x ,\  \_ )$ & First\\
        & $|$ & $(\_ ,\ x )$ & Second \\
        & $|$ & $ x++$ & Increment \\
        & $|$ & $ x--$ & Decrement \\
        & $|$ & $ \textbf{not}\ e$ & \\
        \multicolumn{4}{l}{\emph{Communication Channels:}}\\
        $channel\ $& $::$ &$\textbf{channel}\ x$& Typed channel, i.e. incoming and outgoing end for variable x\\
        & $|$ &$x\textbf{\_receiver}$& Typed receiving end of a  channel  \\
        & $|$ &$x\textbf{\_sender}$ &Typed sending end of a channel  \\
    \end{tabular}%
    }
    \caption{The terms of Ohuas backend language used to represent the DFG. Language specific backend integrations translate this language to generate the output program.}
    \label{tab:DFGdef}
\end{table}

A major advantage of compiling with Ohua is that the generated dataflow-based language is deterministic. That is, a correct, deterministic, sequential program becomes a correct, deterministic, concurrent program by compilation. Formal verification of the compiler transformations to proof this claim and formalize the programming model is currently an ongoing task. Nevertheless, we can already clearly describe the assumptions concerning the concrete implementations of nodes, edges and the runtime each architecture must provide.\\

Specifically, we assume for the implementation of nodes and runtime that:
\begin{enumerate}
    \item Nodes do not share mutable memory. However, the architecture provides access to environment references, which may be global constants, imports, and (most importantly) the arguments of the compiled algorithm.
    \item There can be more nodes than the runtime is capable of running concurrently and there is no explicit scheduling. Therefore we assume cooperative multitasking, i.e. nodes waiting for input will free computation resources for other nodes.
    \item The runtime instantiating the nodes is capable of ending them and freeing resources.
\end{enumerate}

Since it is a data flow language, the execution of the programs is controlled by the data flow. This results in the following assumptions for the implementation of the edges: 

\begin{itemize}
    \item[i)] all data are transferred in order,
    \item[ii)] there is no implicit use of default arguments (default arguments are in general possible, but there has to be an explicit signal for every computation in a node and every parameter it uses that this parameter is 'None' and should be replaced by the default value for this particular execution round/loop),
    \item[iii)] receiving is blocking (this closely relates to ii) in the sense that there must not be a calculation or result passed on while it is not clear whether a term of that calculation just has not arrived at the moment of calculation),
    \item[iv)] all types visible in the compile scope are sendable in the given architecture.
\end{itemize}

Contrary to the assumptions on the input code, the assumptions about the architecture and backend integration cannot be validated inside the compiler. 
    
\section{Micro- and Unikernels}

General-purpose operating system have to provide a broad range of functionalities to connect application layer to hardware layer. This includes user interfaces (graphics), networking, security, device drivers, and most obviously the functionality required for program execution and memory management, which may also include virtualization mechanisms for programming languages as Java and Python. To enable efficient execution and communication between the components, Unix-like operating systems, for example, are often implemented as monolithic kernels. In monolithic kernels the system services (daemons) and drivers have direct access to the hardware and the shared memory. \\

However, this approach has considerable downsides. Since drivers and daemons run in kernel mode, they have full rights and access to the resources of all other processes. This means that in the event of a vulnerability in one of the components, the entire system is affected in principle. In particular third-party device drivers were notoriously faulty and a portal for exploits\footnote{Recent example of \href{https://nakedsecurity.sophos.com/2021/03/17/serious-security-the-linux-kernel-bugs-that-surfaced-after-15-years/}{driver bugs} in the Linux kernel }. Also, many applications do not require most of the services provided by those general purpose systems. For example, a microservice that merely answers simple requests to a key-value store only needs the functionality of the network stack and the file system. Libraries and system functions for additional user or device interfaces only unnecessarily increase the complexity, memory consumption and attack surface of the service. \\

Two alternative concepts of kernels are micro- and unikernels. Figure~\ref{fig:kernels} shows how user applications, drivers and system services, and the hardware layer are compartmentalized in each of these kernel architectures in principle. Basically, the idea of microkernels is to reduce the code run in kernel mode to the absolute minimum required to access the actual hardware layer, while unikernels are often based on a microkernel, but also give non-essential components required for a specific app to run direct access to hardware resources. We will briefly introduce the two concepts, as well as the related concept of library operating systems here.

\begin{figure}[H]
    \centering
    \includegraphics[scale= 0.36]{figures/kernels.png}
    \caption{Structural comparison of Monolithic, Micro- and Unikernels }
    \label{fig:kernels}
\end{figure}

\textbf{Microkernels$\blacktriangleright$} Microkernels follow a \textit{minimality principle} formulated by Liedtke~\cite{jochen1995mu} as \\

\emph{''More precisely, a concept is tolerated inside the $\mu$-kernel only if moving it outside the kernel, i.e. permitting competing implementations, would prevent the implementation of the system's required functionality.''}\\

The central motivation for microkernels is the reduction of privileged, low-level code executing in kernel mode. Based on the insight that large code bases of monolithic kernels come at the cost of a large potential for bugs in privileged, low-level (and therefore hard to check) code, concepts to minimize kernels (and therefore attack surface) date back to the 70$^th$ \cite{hansen1970nucleus}. Concepts like the Mach kernel \cite{accetta1986mach}, separation kernels\cite{rushby1981design}, or isolation kernels \cite{whitaker2002scale} where developed to minimize and isolate kernel space code, to increase security, and enable verification.  \\

The minimality principle basically limits the essential components of kernel-to-memory management (i.e. providing access and access control to address spaces), CPU allocation (i.e. providing access to the CPU in any form of process or thread abstraction and scheduling), and inter-process communication (IPC). Other services, such as I/O, device drivers, networking, and others are run as userland processes, although there might be further distinctions from actual user processes. In addition to the advantage of the smaller attack surface, the low memory requirement also makes microkernels advantageous, especially for embedded systems. \\

This design comes with an inherent performance penalty. In a monolithic system a userspace application requiring access to hardware or system services would cause a single context switch to kernel mode. The request would be answered and the result returned to the userspace process. In a microkernel, however, the request of the user application will be forwarded by the kernel via IPC, e.g. to a driver process that again answers via IPC indirected through the kernel. In this simple scenario, the number of context switches doubles from two to four. Also, the communication among system services is just function calls in monoliths, while it again involves IPC and four context switches for each invocation among services. \\

Currently existing examples of microkernels are L4 (formerly L3 \cite{liedtke1993persistent}), Minix \cite{herder2006minix}, Singularity \cite{hunt2005overview} or the QNX microkernel OS\cite{hildebrand1992architectural} used in embedded systems for example in phones, or as real time OSs in cars.

\textbf{Unikernels\cite{madhavapeddy2014unikernels} $\blacktriangleright$}: Unikernels also tackle the problem of large code base and attack surface in monolithic kernels. However, they follow a different approach concerning process isolation. The 'uni' in unikernels refers to the idea of compiling a specific kernel for each application or even component of larger applications. Basis for compilation is a library operating system written in rather high level languages, the user program, and a configuration file to specify the target architecture and the required library components. Library operating systems provide functionalities of monolithic kernels as independent library implementations. For example, device drivers for physical NICs are implemented in libraries that can be combined with potentially different implementations of the TCP/IP stack. Examples of library operating systems are  MirageOS\cite{madhavapeddy2013unikernels}, Graphene~\cite{tsai2014Graphene}, IncludeOS~\cite{bratterud2015includeos} or Unikraft~\cite{kuenzer2021unikraft}.
In the compiled unikernel, all processes run with kernel privileges and have direct access to the hardware or hardware abstraction layer (e.g. a hypervisor). This reduces size and attack surface compared to monolithic kernels and has, unlike monolithic and microkernels, no IPC overhead for context switches. 
It is not well suited and not intended to be used for multi-user scenarios. However, large applications can be realized with unikernels by distributing the app components into several distinct unikernels. In cloud applications, this setup allows the hypervisor to scale only required parts of the application. 

\todo[inline]{What kind of organization is unikernel.org? \cite{unikernelorg}}
Examples of this are CubicleOS~\cite{sartakov2021cubicleos}, FlexOS~\cite{lefeuvre2021flexos} and M$^3$x~\cite{Asmussen:M3x}, M$^3$v~\cite{Asmussen:M3v}.
\todo[inline]{read \cite{madhavapeddy2014unikernels},CubicleOS~\cite{sartakov2021cubicleos}, FlexOS~\cite{lefeuvre2021flexos} and M$^3$x~\cite{Asmussen:M3x}, M$^3$v~\cite{Asmussen:M3v} \means how are the kernels compiled, what's there scope? }


\section{The M\textsuperscript{3} Operating System}
\subsection{The Concept}
M$^3$\cite{Asmussen:M3x} is a microkernel concept/architecture for distributed and potentially heterogeneous architecture, as for example different embedded processors cooperating in modern cars. It comprises a hardware and a corresponding software, i.e. OS and kernel design. Specifically, the hardware design describes the components necessary to connect and control separate chips, e.g. for broadband communication, signal processing (camera, GPS), or cryptographic operations, while the corresponding operating system provides communication and access to the hardware for the apps running on each component.

An overview of the principle design of M$^3$ is shown in Figure~\ref{fig:m3}. The architecture is composed of tiles, communicating with each other via data transfer units (DTU). There is only one tile that runs the actual microkernel. This tile is also the only one requiring a general purpose core (GPC) as underlying hardware. The computing units (CU) inside the other tiles might be general purpose CPUs, FPGAs, DSPs or fixed-function accelerators. Processes on CUs run independently and isolated from each other. They can however communicate to each other via the DTUs.

The kernel is responsible for scheduling tasks on the CUs. Running tasks are called activities. On tiles using CPUs, an activity is basically a running system thread. Via the DTUs, the kernel can also control context switches between activities on the CUs and establish communication relations between activities. By default, activities run in their own address space and are disconnected from each other. To establish a connection, an activity \code{A} would once request the kernel to connect, for instance to an activity \code{B} running the network stack. Once that connection is established, the activities can directly communicate without involving the kernel again, which eliminates some of the communication overhead in other microkernel systems. 

\begin{figure}[H]
    \centering
    \includegraphics[scale= 0.7]{figures/m3.png}
    \caption{The m3 architecture is composed of tiles, communicating via data transfer units (DTUs).}
    \label{fig:m3}
\end{figure}

In Section~\ref{subsec:BackendRequirements} we discussed the assumptions Ohua makes about the backend architectures' tasks and channels. In particular, we noticed that tasks have to be cooperatively scheduled and channels have to ensure in-order, guaranteed delivery. Here we take a look if these requirements are fulfilled by M$^33$.\\

When activities are idle, they notify the kernel, which can then schedule another activity. So cooperative scheduling is given. Communication channels are unidirectional, first-in-first-out connections. Activities are not aware if their communication partner is currently running or suspended i.e. if the context was switched by the kernel. The hardware (i.e. the DTU of the receiving tile) detects attempts to communicate with a not-running activity and errors back to the sending activity. Upon such error, the sending activity invokes the kernel to schedule the required receiving activity. The kernel buffers messages for the receiver in this case. Despite this fallback mechanism, the authors state that message delivery has only-once semantic, so data access can/must be repeated if necessary upon failure. 

\subsection{The Rust API}
M3 provides a Rust integration to access its abstractions of processes and channels. Here we briefly describe the main features of this API. A simple example of how an activity can be instantiated is shown in Figure~\ref{fig:startingActivity}\footnote{Example adapted from \href{https://github.com/Barkhausen-Institut/M3/blob/master/src/apps/rustunittests/src/tactivity.rs}{M3 Rust unit tests}}. Activities can be initiated either on a common tile or on several tiles. In the example, the tile of the parent process is used. The API uses a closure syntax to define activities. However, those definitions have no closure semantic, i.e. they do not enclose definitions from the surrounding scope.\\

To pass capabilities and file descriptors to a child activity, the attribute \rust{activity.data_sink()} can be used to serialize data into the process memory. This data can be accessed from inside the activity code using \rust{data_source()}. Serialization is implemented in a custom \rust{Serializer} based on the \rust{serde} crate. We need this mechanism, as shown in the example, to pass send and receive gates to the activities. For simple environmental variables, the API provides a separate mechanism using \rust{m3::env}. This basically provides a global key-value store in which variables can be stored before the definition of an activity, retrieved by key inside the activity, and deleted after the activity definition to not pollute the namespace.\\


\begin{figure}
    \centering
    \begin{minted}[fontsize=\footnotesize]{rust}
fn run_send_receive(t: &mut dyn WvTester) {
    // Get a descriptor of the current tile
    let tile = Tile::get("clone|own");
    // Initialize a new activity on the current tile
    let mut activity = ChildActivity::new_with(tile, ActivityArgs::new("test"));

    // initialize send and receive gate with message order, size and credits 
    let rgate = RecvGate::new(math::next_log2(256), math::next_log2(256));
    let sgate = SendGate::new_with(SGateArgs::new(&rgate).credits(1));

    // make receive gate avaible in the activities namespace
    activity.delegate_obj(rgate.sel());
    let mut dst = activity.data_sink();
    dst.push(rgate.sel();

    // define activity 
    let activity = activity.run(|| {
        ...
        // make receive gate available inside activity
        let mut src = Activity::own().data_source();
        let rg_sel: Selector = src.pop().unwrap();
        let rgate = RecvGate::new_bind(rg_sel);

        // receive 
        let mut res = recv_msg(&rgate));
        let i1 = res.pop::<u32>();
        let i2 = res.pop::<u32>();
    });

    send_vmsg!(&sgate, RecvGate::def(), 42, 23));
}
    
    \end{minted}
    \caption{Example of creating and running a Rust activity on M3}
    \label{fig:startingActivity}
\end{figure}

Concerning process communication, the API provides different mechanisms (mainly depending on the size of the data to be transmitted) of which direct sending over channels is the most relevant to us. M3 manages communication among processes using capabilities. The capability to directly send to or receive from another process is implemented as \rust{Gate}s in the M3 Rust API. Gates are synchronous, directed one-to-one connections, so there are send gates \rust{SGate} and receive gates \rust{RGate}. Receive gates are instantiated with a maximum message size and a maximum overall size of the message buffer to prevent memory overflows in limited environments. The absolute (system immanent) maximum message size is 2 kb. Send gates are instantiated with a number of credits. With each message send, one credit is used. The receiver can return those credits by answering on a received message. Note that this mechanism is hidden inside the \rust{recv_msg} and \rust{send_vmsg!} calls. Both are or contain macro calls to generate and receive responses upon message receipt to pass back sending credits. \\

Apart from the credit system, sending messages is straight forward, as known from most pipe-like interfaces. Receiving is done in two steps. First a receive stream is requested using \\
\rust{let stream = rgate.recv_msg()}. This is a blocking call, that will return upon available messages arriving at the gate. The second step is calling \rust{let msg = stream.pop()}, which is non blocking. This call can be done arbitrarily often on an existing stream but will fail if there are no more messages in the stream.\\

Finally, a limitation of the current M3 API is that the standard library is not supported. There is a separate implementation of \rust{std} with essential functions but partly different function signatures. For smoltcp itself this limitation is not relevant, since it is written without using the standard library. In contrast, the libc API is a necessary part of smoltcp and is supported by M3. 

\subsection{Existing M3 Architecture Integration}
Ohua already features an architecture integration for Rust on M3. It is built on a simplified API of M3 that provides two main encapsulations, the \rust{channel()} call and the \rust{activity!} macro. Figure~\ref{fig:OhuaM3Wrapper} shows how an activity can be created using those functionalities. The function \rust{channel} basically wraps the initialization of pairs of send and receive gates with default message orderings, a default message size, and a default of one send credit. This allows to create channels basically the same way as in pure Rust or Python. The \rust{activity!} generates code to 1) instantiate a \rust{ChildActivity} on the current tile, 2) delegate the given gates to the activity, and 3) bind and activate the gates inside the activity. This again basically resembles the initialization of a thread in pure Rust. 


\begin{figure}
    \centering
    \begin{minted}[fontsize=\footnotesize]{rust}
use funs::hello_world;

fn test() -> String {
   use m3::com::channel::{Sender, Receiver};
   use m3::activity;
   let (a_0_0_tx, mut a_0_0_rx) = channel(); 
    activity!(
        (|a_0_0_child_tx: Sender| {    
            let a_0_0 = hello_world();
            a_0_0_child_tx.send(a_0_0)?;
            Ok(())
          }
        )(a_0_0_tx)
    );
    a_0_0_rx.activate()?;
    a_0_0_rx
    .recv::<String,>()
    .expect("Error message")
}

    \end{minted}
    \caption{Example of creating an activity using the simplified M3 Rust API}
    \label{fig:OhuaM3Wrapper}
\end{figure}

However, as previously explained, activities in M3 are not closures. This means that in order to use environment variables, an additional mechanism will be necessary. This is currently not part of the architecture integration.

\section{smolTCP}
smolTCP~\cite{smolTCP} is a Rust-based open-source implementation of a TCP/IP stack. It runs entirely as a user space application. It also provides conditional compilation features to build applications without heap allocation. This makes smoltcp and applications built with it amenable to be used in microkernels such as M$^3$\cite{Asmussen:M3v} and embedded systems such as ARTIQ (e.g. \cite{lam2021combining}). In fact, the microkernel operating system Redox\cite{redoxwebsite}, as well as M3 itself, already use smoltcp for their network stack implementation. So in both systems, smoltcp is used to create a network stack as a sequential, userspace service. This means that, in contrast to the application to be developed in this thesis -- the TCP/IP layer, the actual network layer or system interface and the user application communicate with each other via shared memory. \\

The design of the library is structured according to typical TCP/IP layering. We will briefly introduce the three main layers or components, respectively, that are relevant in this work\footnote{For more information see the documentation at \url{https://docs.rs/smoltcp/latest/smoltcp/}.}.


The \rust{socket} module provides different socket abstractions implementing the TCP, UDP, IGMP or DHCP protocols, respectively, as well as for raw sockets. Common features of all those abstractions are keeping track of inbound and outbound data in socket buffers and implementing functionality to package or unpack those data according to the implemented protocol. The sockets keep also track of additional state information, relevant for their respective protocol e.g. TCP or DHCP client states, hop limits, window sizes, etc.\\

The \rust{iface} module provides the abstractions of the IP layer. The most important structure is \rust{Interface}. In an inner component of the interface (\rust{InterfaceInner}), the state data of the IP layer are stored. This includes the IP address of the interface, a routing table, a neighbor cache, and the hardware address (depending on the transport medium according to Ethernet or IEEE 802.15.4 standard). Accordingly, the generation and interpretation of IP headers for outgoing and incoming packets are also tasks of the 
\rust{InterfaceInner}. In the currently official variant of smoltcp, the Interface also contains a field \rust{Device}, holding an abstraction of the physical network layer, and a \rust{SocketSet} to manage all sockets belonging to the interface. Smoltcp is under ongoing development and so, in the recent implementation state, \rust{Device} and \rust{Sockets} are no longer part of the interface, but independent structures passed to it to process packets from sockets to device and vice-versa. We will discuss the implication of this in further detail in the Section~\ref{sec:ImplSmoltcp}. \\

Finally, the physical layer is implemented in the \rust{phy} module. It provides different implementations of the \rust{Device} trait, to connect the application to the underlying operating systems loopback or tuntap interface or raw sockets. Implementors of the \rust{Device} trait provide the methods \rust{receive}, \rust{transmit} and \rust{capabilities}.\\

The actual transfer of packets from the interface to the device is realized via sending and receiving tokens. We explain this in a little more detail here, because it exemplifies a characteristic of the smoltcp code. A successful call to \rust{device.receive()} will yield a tuple of a receive token \rust{RxToken} and a send token \rust{TxToken}. The former will contain the actual received content as a private field, the latter contains a reference to the device specific storage for outgoing packets. In case of tuntap and rawsocket devices, this reference is a file pointer provided by the operating system. In Figure~\ref{fig:TokensAndClosures} the implementation of a sending \rust{TxToken} for raw sockets and the usage of such token to send an Ethernet frame are shown. Two things become apparent. First, the memory for the packets is allocated only at device level upon consuming a token. Second, any structs needed to construct the frame are instantiated in a closure passed as an argument to \rust{tx_token.consume()}. As closures are realized via fixed size structs on the stack frame of the called function, this technique enables smoltcp to work without heap allocation and is used heavily in the code. This principle is used in a cascading manner i.e. \rust{dispatch_ethernet} itself also receives a closure capturing objects from the calling scope.\\

Obviously, one implication of separating the layers of smoltcp to different, memory separated components is that this kind of memory efficiency cannot be maintained. 

\begin{figure}[H]
\centering
\tabskip=0pt
\valign{#\cr
    \hbox{%
    \begin{subfigure}{.37\textwidth}
    \centering
     \begin{minted}[fontsize=\tiny]{rust}
// Sending Token for RawSocket

pub struct TxToken {
    lower: Rc<RefCell<sys::RawSocketDesc>>,
}

impl phy::TxToken for TxToken {
    fn consume<R, F>(
    self, 
    ..., 
    f: F) -> Result<R>
    where
        F: FnOnce(&mut [u8]) -> Result<R>,
    {
        let mut lower = self.lower.borrow_mut();
        let mut buffer = vec![0; len];
        let result = f(&mut buffer);
        match lower.send(&buffer[..]) {
            Ok(_) => result,
            // Error handling 
        }
    }
}
     \end{minted}
    \end{subfigure}%
  }
  \cr
  \noalign{\hfill}
    \hbox{%
    \begin{subfigure}{.62\textwidth}
    \centering
    \begin{minted}[fontsize=\tiny]{rust}
    //Interface using a TxToken to send a frame
    
    pub fn dispatch_ethernet<Tx, F>(
    &mut self, 
    tx_token: Tx, 
    buffer_len: usize, 
    f: F) -> Result<()>
    where
        Tx: TxToken,
        F: FnOnce(EthernetFrame<&mut [u8]>),
    {
        let tx_len = EthernetFrame::<&[u8]>::buffer_len(buffer_len);
        tx_token.consume(self.now, tx_len, |tx_buffer| {
            ...
            let mut frame = EthernetFrame::new_unchecked(tx_buffer);
            let src_addr = {...};
            
            // closure from outer scope:
            f(frame);
            Ok(())
        })
    }
    \end{minted}
    \end{subfigure}%
  }
  \vfill
  \cr
}
\caption{Sending and receiving packages is implemented via Tokens exposing a \rust{consume} method, taking closures as argument that process the send or received content. This way, memory allocation can be constrained a) to the device layer and b) to the stack if necessary.}
\label{fig:TokensAndClosures}
\end{figure}

\section{Rust}
\label{subsec:Rust}
The main questions of this thesis are i) What requirements must a sequential program meet in order to be converted into a concurrent program in a semantics-preserving way? and ii) What steps are necessary for the conversion? Rust as a programming language is particularly well suited to investigate these questions. This is due to Rust's extended type system that cannot only enforce well-typedness of a program but also statically ensure the validity of references and therefor safety of memory usage at runtime. The two central concepts enabling this are \emph{ownership} and \emph{lifetimes}. \emph{Ownership} is used to ensure memory safety in (safe) Rust. In particular, it allows the Rust compiler to ensure at compile time that no two pieces of code can modify an object at any time, i.e. that at runtime there is never more than one reference to an object that allows write access. \emph{Lifetimes} extend the concept of ownership to track the validity of references. This model does not only prevent frequent errors in sequential programs, it also enforces already a substantial part of the assumptions of a distributed programming model. Therefore, necessary restrictions of a distributed, concurrent programming model are particularly well to be examined in Rust, because they are already enforced in the sequential code. In Chapter~\ref{Chapter:Implementation}, we will refer to those concepts and their implications for our findings. So to better understand these implications, we will first take a closer look at the rules of ownership and lifetimes here.\\

\textbf{Ownership:} Rust has neither garbage collection nor does the user need to free and allocate memory herself. Instead, Rust's runtime manages memory using ownership and lifetime rules. The Rust compiler ensures, for most types statically, that there is exactly one \emph{owner} of each value, i.e. one variable holding it, created either on heap or stack. When this owner goes out of scope (i.e. when the scope it has been declared in ends), the value is cleared from memory without the need for extensive reference tracking garbage collection.
This is realized by automatic implementations of the \rust{drop} function for every type. A call to \rust{drop} is added automatically by the compiler for each variable as it goes out of scope.\\

Ownership has a different implementation for values located on the stack and on the heap. For values stored on the stack, the owner is simply the variable assigned to that value. For a value allocated on the heap, the variable owning that value holds a pointer to the memory location of that value and additional information as for instance actual and totally allocated size of the value. This difference is important for the semantics of passing values as function arguments or returning them from functions. The act of passing values directly, i.e. not only passing a reference, is called \emph{moving} in Rust parlance. For both the stack based and the heap based values, \emph{moving} means copying the owner, i.e. copying the information the owner holds on the stack, to the stack frame of the called function. Returning values also follows that principle. But, as we noticed before, for stack based values this actually also copies the value itself. This results in a new value, with a new owner, and allows the old owner to remain valid. For heap based values on the other hand, only the memory reference to the value is copied that way. So, to prevent having several owners addressing the same memory location, the old owner gets invalidated upon \emph{moving}. To actually duplicate heap data, one needs to implement or derive the \rust{Clone} trait for the data type. An explicit call to the \rust{clone()} method will then deepcopy the heap allocated data and its owner, resembling the semantics of implicit copy on stack based values.\\

In a distributed scenario, passing a value to a function call will involve transferring that value to another process's memory that we do not expect to sync with the original location. So, as it is enforced by the Rust compiler, we cannot use the original reference to that value any more, since it would lead to inconsistent states. In fact, Ohua's programming model, at least theoretically, is more strict than that of Rust. While it always assumes pure function arguments to be used immutably, i.e. read-only, it also currently requires variables to be used only once. \\

\textbf{References:} Obviously, besides passing the value directly, Rust also offers the possibility of creating references to both stack and heap based values. In contrast to pointers, Rust's references are guaranteed to address valid data. References do not convey ownership. In Rust's memory model this means, references are just an address to the owner, and do not contain information about the size and capacity of the data pointed to. Passing values by reference is called \emph{borrowing} in Rust terminology. References, just like values, are immutable by default. The general rule for borrowing a value is that at each point in the code there can only be either one mutable or an arbitrary number of immutable references to each owner. References are values themselves, i.e. they have an owner and a lifetime tied to the scope they are valid in. So Rust can track validity of reference owners to enforce the borrowing rules. Beyond scope validity, the rust compiler is capable of identifying the last use of a reference and shorten its lifetime to this point (a concept called 'non-lexical lifetimes'). Therefore it is possible to borrow a value mutably multiple times within a scope. When one mutable reference is created and used only after the last use of another one, the compiler can shorten the former one's lifetime to end before the next reference becomes valid. \\

\textbf{Lifetimes}\footnote{We will focus on the main aspects for this work here, but a full introduction to Lifetimes, as well as rules for subtyping and inference can be found in the Rust Book\cite{rustbook} and the Rust Reference\cite{rustreference} in the Chapter \emph{Subtyping and Variance}.}: As explained before, references do not convey ownership. So \emph{lifetimes} are Rust's way to ensure that the owner a reference is pointing to is not dropped while the reference is still in use. 
In general, every reference has a lifetime. Inside a single function scope, lifetimes are mostly implicit. The compiler can simply derive them from the local scoping. However, when references are used as an argument or return value, lifetimes cannot in general be derived by the compiler and need to be made explicit. Note that the constructors of \rust{enum} and \rust{struct} types are essentially also just functions, taking fields as arguments and returning the respective \rust{emun} or \rust{struct}. So obviously they also have to abide lifetime rules. \\

As we noticed, lifetimes are meant to ensure reference validity. The mechanism is best explained using the case of function return values. Returning valid values, either by value or by reference from a function is only possible if the owner of that value will not be dropped at the end of the function. This is the case if 
\begin{enumerate}[a)]
    \item The function returns a moved owner: The returned value is allocated on the heap and the actual owner is returned. This will result in a move, copying the owners reference information to the return stack frame.
    \item The functions returns a copied owner: The returned value is allocated on the stack, but implements \rust{Copy}. This will copy the value itself to the return stack frame.
    \item The function returns a reference to the owner: The actual owner of the value lives in the outer scope and is manipulated by reference in the function. If a reference to the original owner is returned, the owner will continue to be mutably borrowed for the lifetime of this reference. This also includes references to static memory, e.g. string literals known at compile time. 
\end{enumerate}

When are lifetime annotations needed? The simple answer is: whenever there are references and the compiler cannot infer them automatically. The first two cases generally do not require lifetime annotations. In case a) the data live on the heap, but the 'owning information' is copied to the return scope so Rustc can automatically infer its lifetime further on by tracking the owner. In case b) the data is copied to the stack of the return scope and again the compiler can automatically infer scopes and lifetimes. In case c), however, the lifetime of the reference returned is bound to the lifetime of the memory location (reference) passed as a parameter. The reason why not every function accepting and/or returning references needs lifetime annotations in parameters, return value, and the function itself is the so-called 'lifetime elision'. Rust uses a set of rules to automatically annotate lifetimes in standard situations, which are
\begin{enumerate}
    \item In a function declaration all references are assigned a freshly generated lifetime parameter\\
    \rust{fn fun<'a, 'b, ..., 'z>(a:&'a A, b:&'b B, ...) -> &'z Z}
    \item If there is only one input reference, its lifetime is also assigned to the output reference.\\
     \rust{fn fun<'a>(a:&'a A) -> &'a Z}
    \item If the function is a method all outputs are assigned the lifetime of the object acted on.\\ \rust{fn fun<'self, 'a, ... >(& 'self self, a:&'a A,...) -> (&'self Z1, &'self Z2} 
\end{enumerate}

The same lifetime rules that hold for function arguments and returns basically hold for structs. A \rust{struct} holding a reference in a field requires that reference to come from a surrounding scope. Just as a function cannot take a reference that might not live as long as the function runs, a \rust{struct} can only live as long as the shortest living reference it holds. The same holds for \rust{enum}s, obviously. Just like for functions, this means that those \rust{struct}s and \rust{enum}s are generic over at least one lifetime parameter, determined by the reference they are given on creation.\\

When those rules are insufficient, annotations are needed. For example if there are multiple input references and a specific one should provide the lifetime information for the output. Also automatic inference might lead to unwanted effects. For instance, if a reference to a \rust{static} value is passed to a function, the compiler will infer the returned reference to also live for \rust{static}. This means the original value cannot be borrowed mutably or even immutable again in the program. In this case lifetime annotations help the compiler to solve lifetime unification at all or in a more favorable way. \\

So when and why would we need to consider lifetimes? We expect the programmer to provide valid input Code to the compilation. Also, the programming model of Ohua prohibits reference arguments in scope. So up to this point, there was no necessity to consider lifetime implications in the transformations Ohua applies to the code. However, in this work we will not only manipulate the code 'in scope' but also inside the components that will later be opaque to Ohua. It is possible, and for efficiency reasons desirable, to keep on working with references inside those components. Meanwhile, we will also introduce and in particular split existing functions. Therefore, we will need to consider constraints and effects of reference lifetimes on our transformations.






\chapter{Implementation}
\label{Chapter:Implementation}
\section{Adaptations to make sequential smoltcp amenable to 'distribution'}
\note{I might mention:
While this work was being conducted, smoltcp was being further developed. In particular, the device became an independent component. This logically moves the structure of the library in the direction we are aiming for. 
Therefore, we decided to use this new state of smoltcp as a further basis for the work and to adapt the previous changes accordingly. 
}

In this section we discuss, how our example application is composed using the original smoltcp API and how we need to transform the application itself, as well as smoltcp to make it amenable to compilation. 


\subsection{The initial situation}
In fact, there are two main aspects that need to be considered. The first is, that we need to present the compiler our target components and their basic interaction as stateful objects in the compile scope. It must be recognizable for the compiler, what the stateful components of the program are and how they interact. In particular, we must not as smoltcp currently does, give the device as a parameter to the TCP/IP component and let the later use the device 'under the hood', i.e. inside the library and outside the compile scope. The second aspect is, that all communication between components must be done explicitly via serializable messages, not via access to shared memory as before.

Figure~\ref{fig:oldTopLevel} shows a simple example of a server application using smoltcp. Obviously, this code does not meet our compilation requirements. Concrete problems are 
\begin{enumerate}
    \item Sockets are logically a part of the TCP/IP stack and should not appear in the compile scope
    \item In the call \rust{iface.poll} the state of an independent component, the device, is passed to the another component, the interface
    \item There are several direct accesses to the internal states of other components, e.g. in \rust{socket.listen(6969)} the application directly accesses a state of the TCP/IP stack
    \item States are not used linearly i.e. \todo[inline]{describe} 
\end{enumerate}


\begin{figure}[H]
    \centering
    
\begin{minted}[fontsize=\small]{rust}
fn main() {

    let mut store = Store::default();

    let mut device = TunTapInterface::new("tap0", Medium::Ethernet).unwrap();
    // ... more interface initialization code
    let mut iface = builder.finalize(&mut device);

    let mut sockets = SocketSet::new(vec![]);
    // .. more socket initialization code
    let tcp_socket = tcp::Socket::new(tcp_rx_buffer, tcp_tx_buffer);
    let tcp_handle = sockets.add(tcp_socket);

    loop {
        let timestamp = Instant::now();
        match iface.poll(timestamp, &mut device, & mut sockets) {
            Ok(_) => {}
            Err(e) => {
                debug!("poll error: {}", e);
            }
        }

        let socket = sockets.get_mut::<tcp::Socket>(tcp_handle);
        if !socket.is_open() {
            socket.listen(6969).unwrap();
        }

        if socket.may_recv() {
            let input = socket.recv(process_octets).unwrap();
            if socket.can_send() && !input.is_empty() {
                debug!(
                    "tcp:6969 send data: {:?}",
                    str::from_utf8(input.as_ref()).unwrap_or("(invalid utf8)")
                );
                let outbytes = store.handle_message(&input);
                socket.send_slice(&outbytes[..]).unwrap();
            }
        } else if socket.may_send() {
            debug!("tcp:6969 close");
            socket.close();
        }

        phy_wait(fd, iface.poll_delay(timestamp, &sockets)).expect("wait error");
    }
}
\end{minted}
    \caption{Simplified example of a server application using smoltcp}
    \label{fig:oldTopLevel}
\end{figure}


\subsection{The Goal}
So to better understand the required code changes we start at the top-level. How does the code we have seen in the previous section need to look like to produce the desired output program. Figure~\ref{fig:newTopLevel} illustrates the code we want to achieve. The first, and most trivial change is encapsulating initialization into \rust{init_<component>()} functions. This also entails encapsulating the sockets and the interface from the original code into one component.
\todo[inline]{refactor exapmle and add description when I know how I handle the socket handles}

Second we notice that each of our components is used only once in our target program. The function \rust{poll} that contains the interaction of  TCP/IP stack and the device to receive and send messages must become a pure function, i.e. taking both the TCP/IP stack and the device as arguments. We must lift it into compile scope and refactor the  usage of both components in the function to also follow our programming model. To enable this linear usage of states, we obviously need to transform the way those stateful components work in smoltcp. In Section~\ref{subsec:StateThreading} we further illustrate the problem and explain how to approach it.

Thirdly we see, that the communication between the components does not happen through mutual access to states anymore. For example, instead of accessing a socket directly via \rust{let socket = sockets.get_mut::<tcp::Socket>(tcp_handle);} we need to implement a data format and a command interface in both the \rust{app} component and the \rust{tcp_ip_stack} to enable such interactions via message passing. We discuss this issue in Section~\ref{subsec:MessagePassing} and present our implementation as a possible solution.

\begin{figure}[H]
    \centering
    
\begin{minted}[fontsize=\small]{rust}
fn main() {
    let mut app = init_app();
    let device = init_device();
    let (mut tcp_ip_stack, socket_handle) = init_tcp_ip_stack(device);
    // Contains: Actual payload messages for the sockets i.e. like {handle:[msgs]} and socket state information
    let socket_proxy_state = init_socket_proxy();
    while true {
        let states_n_data = poll(tcp_ip_stack, device, socket_proxy_state);
        let processed_data:CommandsAndMessages = app.do_your_thing(states_n_data);
        socket_proxy_state.update_all(processed_data) 
    }
}
\end{minted}
    \caption{To enable Ohua to extract a data flow graph with local states we need to lift the envisioned stateful components 'app', 'tcp\_ip\_stack' and 'device' and their interaction into compile scope as shown here.}
    \label{fig:newTopLevel}
\end{figure}

Finally we have to use a recursion instead of a \rust{while} loop. This is a current restriction of Ohua which we discuss in more detail in Section~\ref{subsubsec:WhileLoops}. 
\note{Note to me: Ohua also supports 'endless' for loops. However we have no way of gracefully ending them as we a) can not access the loop generator while running and b) we do not support early returns i.e. \rust{break} or \rust{return}}
\todo[inline]{figure out recursion}



\subsection{Transformations needed for state threading}
\label{subsec:StateThreading}
\todo[inline]{Adapt to better fit in the general flow}

In a sequential, single threaded program we can invoke a function on one 'component A', do something on it's state, call a function on another 'component B' and finish the outer call by maybe changing 'A's state again. This is what happens, when a packet is send or received using smoltcp in its current form. However in a concurrent, distributed data flow graph, we can not realize this bi-directional dependency among stateful components. Why is that? The main problem here is realizing concurrency and state locality at once. We could connect distributed components A and B using blocking connections, i.e. A calls B and blocks until it receives the result. What we achieve this way is a distributed, sequential program. Another way would be to have A send its state along with the call to B and B would return this state along with its answer. So A can process the next input and upon receiving an answer it can resume the appropriate state. However, this contradicts the requirement of state locality. In particular when states are large objects e.g. a database state, a training state of a deep neural network or alike, we want them to stay encapsulated in on component.\\

So to distribute the components of smoltcp, we need to disentangle state dependencies from each other. In the following subsection, we explain how this can be done using the TCP/ipv4 implementation of smoltcp. We will look in particular on how our envisioned components 'application', 'tcp/ip stack' and 'NIC' are connected during packet sending, receiving and finally in the \rust(poll()) call, that connects both. For each of those processes we will discuss how to transform the implementation of smoltcp, to extract a data flow graph using Ohua. 
\todo[inline]{Refer to linear State use here when the according background section is ready}


\subsubsection{Sending a Packet}
Figure~\ref{subfig:egressOrig} depicts a simplified control flow during the sending a package, in particular the \code{TCP/IP} component actually represents a nested structure and function calls. The point is however, that when our envisioned component \code{APP} call our envisioned component \code{TCP/IP} which in turn calls \code{NIC}, the callers have to keep their state and block execution until the callees return. This will not be the case, when we deploy those components as part of a data flow graph. In this case, each of the components will process any input based on its state output a result and wait for the next input. So we basically need the components to 'stream process' data. The second Figure~\ref{subfig:egressTransf} illustrates, how this can be done in our example. The forward pass of the call is basically the same as before, but we introduce conditionals to gate the data flow down the stack. The backward pass returning the result of sending is processed not by returning to the pre-call stack position, but also by explicitly passing a message through the components again. 
\todo[inline]{Describe where the if's belong to, mention order is preserved.}


\begin{figure}[H]
    \centering
    
\begin{minted}[fontsize=\small]{rust}
// interface.rs
fn socket_egress(&mut self) -> Result<bool> {
    /* get references to  device, inner_interface and sockets
       hold by the interface struct
       iterate sockets -> socket
       ask inner_interface if socket can egress
    */
    ...
        
        let mut device_result = Ok(());
        let socket_result = socket.dispatch(
                inner, 
                |inner, resp| {
                    let response = IpPacket::Raw(response);
                    neighbor_addr = Some(response.ip_repr().dst_addr());
                    let tx_token = device.transmit().ok_or(Error::Exhausted)?;
                    device_result = inner.dispatch_ip(tx_token, response);
                    device_result
                    }
            );

        match (device_result, socket_result) {
            // Error handling ...
        }
    // if any packet has been send:
    Ok(emitted_any)
}
\end{minted}
    \caption{Simplified code of the \rust{Interface.socket_egress} method}
    \label{fig:egressCode}
\end{figure}

\begin{figure}[H]
    \centering
\begin{minted}[fontsize=\small]{rust}
// tcpsocket.rs
pub(crate) fn dispatch<F>(&mut self, inner: &mut InterfaceInner, emit: F) -> Result<()>{
    /* check socket state using the inner interface
       maybe reset socket state and/or return an ErrExhausted
    */
    ...
    // build IP representation
    let mut ip_repr = 
    // build TCP representation
    let mut tcp_repr = 
    // set fields of the tcp packet, based on socket state and
    // inner state, adapt ip packet (payload len)
    ...
    // This calls the closure from outer scope
    emit(inner, (ip_repr, tcp_repr))?;
    
    // If this point is reached, emitting succeeded and
    // socket state is updated based on the tcp_repr send e.g.
    ...
    self.remote_last_ack = repr.ack_number;
    
    
}
\end{minted}
    \caption{Simplified code of the \rust{socket.dispatch} method}
    \label{fig:dispatchCode}
\end{figure}

\begin{figure}[H]
\centering
\tabskip=0pt
\valign{#\cr
    \hbox{%
    \begin{subfigure}{.3\textwidth}
    \centering
     \includegraphics[height=5cm]{figures/egress_original_simple.png}
     \caption{Original flow of a packet sending call}
     \label{subfig:egressOrig}
    \end{subfigure}%
  }
  \cr
  \noalign{\hfill}
    \hbox{%
    \begin{subfigure}{.65\textwidth}
    \centering
    \includegraphics[height=5cm]{figures/egress_new_simple.png}
     \caption{Refactored control flow of the egress function call}
     \label{subfig:egressTransf}
    \end{subfigure}%
  }
  \vfill
  \cr
}
\caption{Transformation of the packet sending control flow to achieve state locality}
\label{fig:TransformEgress}
\end{figure}


\subsubsection{Receiving a Packet}

Just like sending, receiving is triggered in the \rust{Interface::poll()} method. In the \rust{Interface.socket_ingress()} method, a while loop receives packages form the device as long as there is a next package available. The main aspects of \rust{Interface.socket_ingress()} are depicted in Figure~\ref{fig:socketIngressCode}, lengthy parts are abbreviated by \rust{/**/}comments describing their effects.
Each packet is then 'unwrapped' from its protocol layers Ethernet and IP by the Inner Interface and redirected to the appropriate socket.  This may change the state of the Interface e.g. adding new sender IP addresses. The socket processes the package, again this is a state changing operation and returns a result that may contain an answer packet. If so, this packet is wrapped into an IP packet and Ethernet Frame by the Interface and forwarded to the device for sending. Otherwise, successful processing of a package just sets the \rust{processed_any} flag to let the calling application know that socket states have changed due to received packets.  

\begin{figure}[H]
    \centering
\begin{minted}[fontsize=\small]{rust}
fn socket_ingress(&mut self) -> bool {
    /*get references to sockets, inner and device*/
    
    // try to receive a data buffer object (rx_token) 
    // and a file handle to the device (tx_token)
    while let Some((rx_token, tx_token)) = device.receive() {
        // rx_token.consume applies the closure to the received data frame
        if let Err(err) = rx_token.consume(inner.now, |frame| match
            inner.caps.medium {
                /* matching for different medium, but we only care for 
                   Ethernet for now
                */
                // process_ethernet: remove protocol layers, direct package to socket
                Medium::Ethernet => match inner.process_ethernet(sockets, &frame) {
                    Ok(response) => {
                        processed_any = true;
                        // maybe socket answers directly with a packet
                        // e.g. with an ACK 
                        if let Some(packet) = response {
                            // package the packet in a frame and send
                            if let Err(err) = inner.dispatch(tx_token, packet) {...}
                        }
                        Ok(())
                    }
                },
                ...
            }) {net_debug!("Failed to consume RX token: {}", err);}
        processed_any
    }
\end{minted}
    \caption{Simplified code of the \rust{Interface.socket_ingress()} method}
    \label{fig:socketIngressCode}
\end{figure}

\todo[inline]{Describe While Loop: I can extend this to the appropriate 'while construct' by using the first case distinction as the while condition. What I still need to account for is the "processes any" flag. It can be transported along from TCP/IP to NIC to APP inside the NICCommand: MaybeSend() and the NICResult: Send().}
\todo[inline]{Where are Errors handled? 
\means All errors are caught and handled inside the loop i.e. it is sufficient to $OR$ the NICResult : Send() in every round. }
\todo[inline]{Actually there is another if-split between TCP/IP deciding whether to loop again or to go to the NIC for sending.}

\begin{figure}[H]
\centering
\tabskip=0pt
\valign{#\cr
    \hbox{%
    \begin{subfigure}{.3\textwidth}
    \centering
     \includegraphics[height=5cm]{figures/ingress_simple_original.png}
     \caption{Original flow of a packet receiving call}
     \label{subfig:ingressOrig}
    \end{subfigure}%
  }
  \cr
  \noalign{\hfill}
    \hbox{%
    \begin{subfigure}{.65\textwidth}
    \centering
    \includegraphics[height=5cm]{figures/ingress_simple_transformed.png}
     \caption{Refactored control flow of the ingress function call. Figure~\ref{fig:messageEnums} shows the new data types used to annotate the arrows.}
     \label{subfig:ingressTransf}
    \end{subfigure}%
  }
  \vfill
  \cr
}
\caption{Transformation of the packet receiving control flow to achieve state locality}
\label{fig:TransformIngress}
\end{figure}


\begin{figure}[H]
    \centering
\begin{minted}[fontsize=\small]{rust}
enum NICResult {
	Send(Result<T , Err>),
	Received(Result<Data, Err>)
} 

enum ToOrFrom {
	To(NICCommand),
	From(NICResult)
} 

enum NICCommand {
	Receive,
	MaybeSend(Option<Package>)
} 
\end{minted}
    \caption{Possible implementations for message formats among the components}
    \label{fig:messageEnums}
\end{figure}


\subsection{Transformations needed for Message Passing}
\label{subsec:MessagePassing}
As explained before the nodes of the DFG in the compiled program cannot access shared memory. So we need to extract the communication relations that rely on direct state acccess and transform them into message passing communication. For example, a server application (later application component), can directly create sockets (later network stack component), query their state and change it. \\


How to
\begin{enumerate}
    \item Zugriffe identifizieren ... im Zweifel erstmal für unsere Beispielanwendung
    \item Identifier für die Komponenten des Zustands (Sockets, inner interface, whatever) einführen, die serialisiert werden k\"onnen und eine gemeinsame Sprache der kommunizierenden Komponenten erm\"oglichen 
    \item Aus allen identifizierten Interactionen ein Nachrichtenformat entwickeln um diese abzubilden \means einfachste Variante ist serialisierung der funktionsaufrufe
    \item Das "Zustands-kapselungs-paradigma" erzwingt soweit ich das sehe eine zentrale "Schnittstelle" pro Komponente, die die eingehenden Nachrichten verarbeitet
    \item Wenn ferig \means kann man das Vorgehen generalisieren/automatisieren
\end{enumerate}


\note{Da die Sockets jetzt nicht mehr Teil des Interfaces sind gibt es mehrere grunds\"atzliche M\"oglichkeiten: 

Ich weiss noch nicht ob es eine automatische/generelle M\"oglichkeit der Ableitung gibt, aber das sieht aus wie Domain-Knowledge \means Programmers Responsibility
}
M\"oglichkeiten:
\begin{itemize}
    \item Man k\"onnte tats\"achlich die Sockets rumschicken, dh. das SocketSet als daten Betrachten 
    \item Man k\"onnte eine vollst\"andige Representation der Sockets auf beiden Seiten haben
    \item app hat eine vereinfachte Version der Sockets mit State und Commands \means wenn man es genau auf eine Anwendung zuschneiden will, k\"onnte man alle \rust{socket.is_something()} Zugriffe identifizieren, die muss der abgeflachte zustand beanworten k\"onnen, und alle \rust{socket.do_something()} Zugriffe identifizieren, die m\"ssen als commands vom Interface verstanden werden. \means Um das sauber zu unterscheidnen muss man den Code der Funktionen untersuchen. Ich weiss nicht ob die Unterscheidung m\"oglich und sinnvoll ist und selbst wenn, ob /wie man z.B. bedingte Zugriffe \rust{if !socket.is_open() {socket.listen(6969).unwrap();}} umsetzen würde. 
\end{itemize}


\textbf{Steps for App-Interface:}
1. The most complex interaction (if not the only one) we need to tackle is the one between the app and the tcp\_ip\_stack. In the new smoltcp implementation, the sockets are a separate entity although they are logically a part of the tcp\_ip\_stack obviously. The app needs to be able to set up and identify sockets to manage connections, map requests and answers and so on. The interface needs the sockets to process incoming and outgoing packets based on their internal state.   




\subsection{Can we generalize?}
Observations
\begin{itemize}
    \item Which components should the program have? \means Could not be deduced from Code \means Domain knowledge of the programmer
    \item How to implement message passing? \means I currently suspect this is also a matter of domain knowledge 
    \item How to transform to linear state usage \means Firstly wee need to know the components. Second we'd need to follow their states in the code. However this probably entails knowing a) which encapsulated object (e.g. socket) belongs to the component and which one doesn't. Assuming the composed states are defined by the programmer we can follow their fields though function calls. We would assume correct code AND no keyword args as in Python so we can follow object using their position in a function call even if they're named differently inside the function \means we need to track aliasing \means this works as long as we can follow all calls i.e. no function calls to outside compile scope takes our objects as arguments \means I have to evaluate against our restrictions if referencing can get our objects can get in our way.
\end{itemize}


\subsection{other stuff}
\todo[inline]{Rewrite with examples}
\begin{itemize}
    \item Panics \means Errors are forwarded, meaning we can react on them in distributed settings, \rust{panic!} can in the M3 scenario at most be detected by the node sending to the panic node because M3 will notify a sender when the receiver is closed. In this scenario, the usual message path through the graph is obviously broken so the only way to propagate this would be an extra channel each node has to the main thread.
    \item currently only tuples of exactly 2 elements are supported \means either I adapt Ohua, or I need to introduce further wrapping for any more complex pattern
    \item currently input to functions i.e. parameters can not directly be used as stateful objects \means I need to introduce wrapper function calls, I might remove them in the backend again.
\end{itemize}

\section{Adaptations to make Ohua fit for relevant applications}
\subsection{Providing full type annotations}
Although Rust is a strongly typed language, type annotations for local variables are often unnecessary, since they are automatically derived from the type inference of the rust compiler. In contrast, definitions of types, i.e. structs, enums and functions, are the basis of type inference and must be annotated manually/by the programmer. Since Ohua produces new code in the backend, it is not enough to transfer existing annotations of the input to the output. In particular, the communication channels of the produced tasks require type annotations, since these cannot be completely derived by the Rust type inference even in the shared memory scenario. Also in view of a later extension to fully distributed systems, distributed compilation of individual components or the use of languages with less powerful inference, it is important not to rely on the type inference specific to the compiled language. 

So where do we get the type information we need? One important observation is that Ohua does not create new types. The control functions Ohua inserts into the data flow graph are mainly there to pass on existing intermediate results of the original program. That is, the types of these functions that is, of their input and output channels can be derived from the input code. For this we need a) a type extraction from the input program and b) a type propagation which propagates the corresponding types through the representation of the data flow graph. Both were basically available at the beginning of the work. However, the existing implementation had problems or was not fully functional. This means that some types had to be annotated by hand in the output code. The following sections describe how type extraction and type propagation worked and which changes were necessary to achieve the desired functionality. \\

\subsubsection{Type Extraction}
So the first functionality we had to address was the \code{Type Extraction} in the frontend integration. Until now this was a two step process. In a first pass two kinds of data were extracted from the input module. One was the algorithms i.e. the Rust functions that were to be compiled. Those were translated to an internal representation of the supported Rust subset as described in Section~\ref{subsec:RustIntegration}. The second structure kept track of imports defined in the module. \\

The second pass was needed to annotate types to function names. As each function call \rust{let z = someFun(x, y);} might become an independent node, the compiler needs type annotations for \rust{x} and \rust{y} to later annotate the channels for sending those variables among independent tasks. To do so first the function names called in the parsed algorithms were extracted. Then all files defined in the imports where scanned for function definitions. From this information, a hashmap was built over all function names and the extracted function types. Finally, in a further traversal over the input code, this hashmap was used to lookup function types and annotate them in the input code. This procedure had some disadvantages, namely : 

\begin{itemize}
    \item The entire compile scope, including for example the standard library, had to be available for the compiler to find and process. This introduced path dependencies of the compiler and notably excludes the import of compiled libraries in other languages e.g. libc, which is critical in our case.
    \item We had to restrict the entire compile scope and imported libraries to syntax the compiler could understand. This has previously been addressed by re-implementing some required libraries in a simpler form. 
    \item We had to keep track of name spaces and aliasing for all functions
    \item We could not support 1. generic type parameters in function definitions and 2. overloaded function definitions.
\end{itemize}

The main point of concern was really the need to parse the whole scope and therefore to have all libraries used available and compatible to Ohua supported syntax. Therefore we chose to change the source of type information. Instead of collecting function signatures from the scope and typing functions globally for the complied module, we now type each call site according to the local context. Remember, we need to type the input parameters for each function call inside an algorithm. Those parameters can be: 

\begin{itemize}
    \item global constants, in which case Rust requires a type annotation
    \item input parameters of the algorithm, in which case we know their type from the algorithm signature
    \item local variables bound in the algorithm, in which case we now require the programmer to provide type annotation 
\end{itemize}

So for most syntax constructs, we can derive the type information needed from the local context. This requires the programmer to annotate types manually in local assignments, where it would not be required by Rust itself. Also Figure~\ref{fig:TypeExtractionExample} shows an example, where additional local bindings are required. In the example code, to be able to type the function call to \rust{h(e)} we need an additional binding statement as it is not possible to type annotate a loop pattern currently. 

\begin{figure}[H]
\centering
\tabskip=0pt
\valign{#\cr
    \hbox{%
    \begin{subfigure}{.35\textwidth}
    \centering
     \begin{minted}[fontsize=\footnotesize]{rust}
fn test(i:i32) ->  {
    let s = State::new_state();
    for e in range_from(i) {
        let r = h(e);
        s.some_method(r);
    }
    s
}
     \end{minted}
    \end{subfigure}%
  }
  \cr
  \noalign{\hfill}
    \hbox{%
    \begin{subfigure}{.62\textwidth}
    \centering
    \begin{minted}[escapeinside=||,fontsize=\footnotesize]{haskell}
fn test(i:i32) -> () {
    let s:State = State::new_state();
    for e in range_from(i) {
        let e1:i32 = e;
        let r:i32 = h(e1);
        s.some_method(r);
    }
    s
}
    \end{minted}
    \end{subfigure}%
  }
  \vfill
  \cr
}
\caption{To extract type information for function call from the local context we require the  programmer to annotate the according types to local variables. As shown in the right code example it is sometimes also necessary to have additional binding statements to annotate every relevant variable i.e. every input to a function call.}
\label{fig:TypeExtractionExample}
\end{figure}

However, the new \code{Type Extraction} works without the need to parse the imported files. As we only type the concrete function call sites, this also allows the use of generically typed.\\
\todo[inline]{Check with felix whether and how this was a problem anyways since Ohua is perfectly fine with generics and also Rustwise it should be ok to infer at compiletime as usual}

The main change, required to implement this solution was threading a monadic context through the complete process of transforming the input code to the frontend representation. In particular also through the first step of this process, where the Rust code is mapped to a subset of Rust supported by Ohua. These context keeps track of variable bindings and according types in the current scope. In the outermost scope i.e. the global level of the input code, this context is pre-filled with constant definitions, including all global constant definitions parsed before the actual algorithms. 

The conversion of algorithms is also monadic process and the initial context contains the names and types extracted from the global scope. For each algorithm the parameter names and their types are parsed from the signature and added to the context. Upon parsing the body of the algorithm each right hand side of a let binding\footnote{We currently only support variable or tuple patterns} is checked for type annotation and registered in the local context if annotated properly. Unannotated bindings will yield an error at this point. AS shown in the example in Figure~\ref{fig:TypeExtractionExample} this might require some additional local assignments, when local variables result from pattern binding.\\

Now, whenever a function call is converted, the function type is derived from its arguments. In the case of a stateful call, this also includes the called object. If the arguments are variables, the types are obtained from the context. If the variables are not in the context, again an error is thrown. For supported literal arguments (currently integers, booleans and strings) the types are derived automatically. Obviously, this limits the accepted input syntax in that only literals and variables are valid as direct function arguments. Still the advantage of being able to use the entire Rust syntax again in the imported libraries outweighs this in our opinion. As function types now depend on the types of local variables we added test cases to the regression test suite to ensure proper typing, when local scope variables shadow names from outer scopes. Notably name shadowing inside the same scope is currently not supported by Ohuas renaming algorithm and could therefore not be tested.\\

\todo[inline]{Check where the below comments belong}
\note{Finally there are some open question we need to address regarding accepted types. Currently the type extraction mechanism just wraps any relevant type annotation from the input code in a to an internal argument type representation. This means, we do not distinguish the actual Rust types and do not filter for references, pointer or trait objects. The goal is of course to develop Ohua and the programming model for Rust to the point where a valid sequential input program never leads to an invalid output program, i.e. we want to achieve soundness. For this we need to evaluate if and how the type system of Rust has to be restricted in the input. And so we need to decide whether to leave this to the programmer, because it depends for example on the chosen backend or if there have to be additional mechanisms in the compiler to enforce these restrictions. We might also need to extend the programming model by necessary augmentation on the original program, like additional trait bounds. For example, an automatic annotation with \rust{[\#derive]} could be generated to add traits, e.g. for serialization, that were not necessary in the serial program. We will consider those question in the further development of Ohua and in particular the Rust integration.}

\subsubsection{Type Propagation}
The second aspect arises in the core compiler. When Ohua generates a Data Flow Graph, it introduces control functions e.g. to guard branching or collect results of a loop (see \ref{} for further details). Some of those functions are only present in intermediate representations because node fusion implemented downstream in the compiler may integrate them into bigger nodes. Some however occur as separate tasks in the final program. In particular for the later kind, proper type annotations are essential but it is sensible to provide them for any such function if possible, to reduce the assumptions among the steps of compilation. Also the transformation of code to SSA form introduces new variable names that need to be typed.\\

Control function by their nature do not introduce new types. So it is possible to infer most of their input and output types from the host-language types parsed in the frontend. The code example in Figure~\ref{fig:TypePropagationExample} shows a Rust function and its last representation in the compilers \code{DFLang} representation. Marked in $\mathbf{bold}$ we can see the functions Ohua introduced to control the dataflow in the final program. We can also see, that one of those functions, namely $\mathbf{smap}$ is not preceded by the namespace marker $\mathbf{ohua-lang/}$. This is because most of the control functions are currently not represented by own constructors of the function representation in \code{DFLang}. They are represented internally just the same as host-language function calls and only recognized in pattern matching upon their names. In contrast $\mathbf{smap}$ is already implemented as a separate constructor. 

\begin{figure}[H]
\centering
\tabskip=0pt
\valign{#\cr
    \hbox{%
    \begin{subfigure}{.35\textwidth}
    \centering
     \begin{minted}[fontsize=\footnotesize]{rust}
 fn test(i:i32) -> () {
    let s:State = State::new_state();
    for e in range_from(i) {
      let e1:i32 = e;
      let r:i32 = h(e1);
      s.gs(r);
    }
 }
     \end{minted}
    \end{subfigure}%
  }
  \cr
  \noalign{\hfill}
    \hbox{%
    \begin{subfigure}{.62\textwidth}
    \centering
    \begin{minted}[escapeinside=||,fontsize=\footnotesize]{haskell}
    let s_0_0_1 = 
        |$\mathbf{ohua.lang/unitFun}$|(State/new_state, ()) in
    let a_0_0 = /range_from ($i) in
    let (d_1, (ctrl_0_0, ctrl_0_1), size_0) = 
        |$\mathbf{smapFun}$|(a_0_0) in
    let s_0_0_1_0 = 
        |$\mathbf{ohua.lang/ctrl}$|(ctrl_0_0, s_0_0_1) in
    let lit_unit_0 = 
        |$\mathbf{ohua.lang/ctrl}$|(ctrl_0_1, ()) in
    let r_0_0_0 = /h (d_1) in
    let (_, ) = /gs [s_0_0_1_0] (r_0_0_0) in
    let d_0_0 = 
         |$\mathbf{ohua.lang/unitFun}$|(ohua.lang/id, lit_unit_0) in
    let x_0_0_0 = 
        |$\mathbf{ohua.lang/collect}$|(size_0, d_0_0) in
    let c_0_0 = |$\mathbf{ohua.lang/seq}$|(x_0_0_0, ()) in
    c_0_0
    \end{minted}
    \end{subfigure}%
  }
  \vfill
  \cr
}
\caption{Example of a Rust input function and its last stage in the core compiler representation DFLang. Functions in $\mathbf{bold}$ are control functions, introduced during compilation that need to be type annotated.}
\label{fig:TypePropagationExample}
\end{figure}


Basically, the \code{Type Propagation} works as follows. Remember in the frontend we annotated the argument types for the called Rust functions, in the example in Figure~\ref{fig:TypePropagationExample} the function calls \rust{State::new_state()}, \rust{range_from(i)}, \rust{h(e1)} and \rust{gs(r)}. Using this information the type propagation happens in a bottom-up traversal over each compiled algorithm. Due to this bottom-up processing each use of a variable is processed before its assignment. That means in our example, the function call \rust{(_, ) = /gs [s_0_0_1_0] (r_0_0_0)} is processed before the assignment \rust{r_0_0_0 = /h (d_1)}. As the function call is processed its argument types are used for two things 1. update the type field of the variables used in the call and 2. update the context to contain the associations between the variable names and the according function type arguments. \\

Contrary to Rust function calls argument types of Ohua control functions are not annotated at this point. However we can always tell their output type from the input, because they only guard data flow and do not calculate results themselves. For example the \code{collect} control function is introduced to collect the results of a loop. We know its signature has to be \code{collect:: nat -> A -> [A]}, to collect a given number of arguments of type \code{A} before returning a list of type \code{[A]}. If the output list is used by another function downstream, we already know the type \code{A} from the context and can completely annotate the variables in statements using \code{collect}.\\

Now there were two problems with the existing implementation of the \code{Type Propagation}. The first rather trivial one was, that several control functions where not or incorrectly processed. The second problem is illustrated in the  code example. It is possible that a graph ends in one or more control functions. In this case there output is not used by any typed Rust function, so we could not type there arguments and returns in the bottom up pass. Due to this problems, the existing \code{Type Propagation} was only partially functional. While the former \code{Type Extraction} mechanism just complicated or restrained the assembly of proper input code, missing \code{Type Propagation} functionality actually leaded to non functional code, in the sense that type annotations had to be made manually in the output code after compilation.  \\

To fix this issue we have made two main changes. First, we have extracted the return type for all algorithms, i.e. all compiled functions in the frontend. This is now passed as an additional parameter through the compiler pipeline and is available in the type extraction. In the example code in Figure~\ref{fig:TypePropagationExample} the return value is \rust{c_0_0}. Given this type information we can now also annotate Ohua control functions whose output is not used by an annotate Rust function in the bottom-up pass. The second change was obviously to fix or add type propagation for previously wrongly typed control functions. Finally all tests where adapted to expect correct typing of the output code. \\

One problem that has not been addressed in this work is that the control functions are not represented by separate constructors in the DFLang. This means that they can only be distinguished from normal Rust functions by their function name, which is error-prone and difficult to maintain. So in the future the control functions should at least be mapped in their own constructors. Also arguments with known type, like the first argument of the function \code{collect:: int -> A -> [A]} should optimally be enforced by the type system. 


\todo[inline]{Describe SSA problem and fix as soon as tests work}


\subsection{Not yet supported Rust Features}
\subsubsection{Error Handling} 
There are different ways of handling errors in Rust that come with different difficulties for us. The actually troublesome cases are those who result in either a panic, or an early return. 
    \begin{enumerate}
        \item matching \rust{Ok} or \rust{Err} \means We will have to handle matching in general, so this is a separate task
        \item handling with combinators as \rust{and_then},  \rust{or}, \rust{or_else} \means some of them just act as monadic functions i.e. they only produce a \rust{Some} from \rust{Some} and a \rust{None} from a \rust{None}. Others can result in early return e.g. \rust{or_err()} and we need to handle this.
        \item Rust uses the \rust{?} symbol as syntactic sugar for what is effectively a try-catch block in other languages \means This is something we need a separate handling for. In particular we need to evaluate if it can be translated into simpler language constructs. This will require more effort if it occurs in sub-scopes of the function block (e.g. inside loops or branches) as it is effectively a return from the function and we do not allow return statements other than at the end of a function. Another culprit is, that it's not just used for Result but also for Option (and maybe others). So effectively it is an early return in the failure case, but we need to deduce what exactly is early returned. 
        \item \rust{unwrap()} \means will panic if called on the 'failure' side of \rust{Option} or \rust{Result}, so this is a matter of 'panic propagation', rather than of error handling
        \item \rust{try!()} \means this is the old version of the \rust{?} operator, so if we find a way to handle the former, we can handle the later in a simpler way because \rust{try!()} is only used for \rust{Result<T, Err>}
        \item \rust{expect("msg")} will early return and error with the specified message \means we need to handle early return here as well
    \end{enumerate}
    
Options to handle this: 
General question is, how do we handle \rust{panic}. As long as a program runs in a single process i.e. pre-compilation a \rust{panic} anywhere during execution will shut down the whole program. This is usually not the case in distributed scenarios as in microkernels, so we need to introduce a way to communicate failure among the separated components and to handle it appropriately.
\todo[inline]{What does M$^3$ do if one of the tiles/one process produces an error? }

So how can we handle the \rust{?}?

Example 1. for outermost function scope:
\begin{minted}[fontsize=\small]{rust}
fn example() -> Result<T,E> {
    let x = might_error()?;
    // do something with x here
}
\end{minted}

\begin{minted}[fontsize=\small]{rust}
fn example() -> Result<T, E> {
    if let Ok(x) = might_error() {
        // do something with x here
    } else {
        // otherwise we'd return the result of the function call
        // to preserve the exact error type
        might_error() 
    }   
}
\end{minted}
$\Rightarrow$ To do this, we need to know the type of \rust{might_error}, because the \rust{?} could also indicate an \rust{Option<T>} that we'd need to handle with  \rust{if let Ok(x) = might_error()}. The problem with this is, that there might be aliases for the Result or Option types so we can not in general know, what the type of \rust{might_error} unless we have all the necessary definitions and aliasing in scope.\\

\bigskip

\subsubsection{Match Expressions}
Although it is slightly counter intuitive as Ohuas Frontend Language is functional and functional languages are notorious for pattern matching, there is no syntax for Rusts \code{match} expressions in Ohuas frontend language. So we either need to transform it into more basic constructs i.e. i-t-e, or encapsulate it in a library function\\


\begin{figure}[H]
\centering
\tabskip=0pt
\valign{#\cr
    \hbox{%
    \begin{subfigure}{.45\textwidth}
    \centering
     \begin{minted}[fontsize=\small]{rust}
// matching in a function
let x = match some_call() {
    pattern1 => something
    pattern2 => something_else
    _ => default_thing
} 
     \end{minted}
     \caption{Example for matching in pseudocode}
     \label{matching}
    \end{subfigure}%
  }
  \cr
  \noalign{\hfill}
    \hbox{%
    \begin{subfigure}{.45\textwidth}
    \centering
    \begin{minted}[fontsize=\small]{rust}
 // replaced function code 
 let x = match_fun();
 //----------------
 // In a new separate library
 fn match_fun() {
    use lib::some_call;
   
    match some_call() {
        pattern1 => something
        pattern2 => something_else
        _ => default_thing
   }        
 }  
    \end{minted}
     \caption{Pseudocode for compilation result}
     \label{matchInLib}
    \end{subfigure}%
  }
  \vfill
  \cr
}
\caption{Match moved to a library function}
\label{fig:MatchHandling}
\end{figure}
\question{Can we rely on Rusts type inference here? i.e. 
If, in the original code \code{x} did not have a type annotation, does this mean that the type of \code{match\_fun()}can also be derived?}
$\Rightarrow$ It is at least not trivially possible to transform a match into an if-then-else because gates in ite are expressions not patterns.
$\Rightarrow$ For some cases, although I'm not sure for how many it is possible to rewrite using if-let as: 

\begin{figure}[H]
\centering
\tabskip=0pt
\valign{#\cr
    \hbox{%
    \begin{subfigure}{.45\textwidth}
    \centering
     \begin{minted}[fontsize=\small]{rust}
let x = match some_call() {
    Some(y) => y
    _ => default_thing
} 
     \end{minted}
    \end{subfigure}%
  }
  \cr
  \noalign{\hfill}
    \hbox{%
    \begin{subfigure}{.45\textwidth}
    \centering
    \begin{minted}[fontsize=\small]{rust}
let x;
if let Some(y) = some_call() {
    x = y
} else {
    x = default_thing
}
    \end{minted}
    \end{subfigure}%
  }
  \vfill
  \cr
}
\caption{Rewrite match using if-let ?}
\label{fig:MatchHandling2}
\end{figure}
\bigskip

Further notes:
\begin{itemize}
    \item inside the arms, there can be \rust{panic!}, \rust{return}, \rust{break} or other kinds of early return. So even if we encapsulate, we need to parse the match to exclude this.
    \item arms of a match expression are expressions. This means it is always possible to assign a match to some value, even if the match was not a right-hand-side in the first place.
    \item the problem with encapsulating matches is, that the arms (can) act on states i.e. we'd have to scan the arms for used states from outer scope 
    \item IF the match was not an assignments right-hand AND we excluded early returns the only sensible function of that match is to manipulate stateful objects. This means, encapsulating does not make sense for such cases because the encapsulating function would take stateful things as arguments and use them as states. Maybe we can handle this by having the encapsulating function return all the (potentially) manipulated states.
    \item[] $\Rightarrow$ We have to do the same 'exclude early return \means collect used states \means make them arguments and return values \means transform' for while loops
\end{itemize}

\subsubsection{Structs and Impl Functions}
\begin{itemize}
    \item Do we need serialization for all of them? How to do this in Rust?
    \item Do we need to preserve the original API in any scenario? If we rewrite structs to tuples for example, calling applications will crash.
\end{itemize}

\subsubsection{While Loops}
\label{subsubsec:WhileLoops}
A very common pattern of server applications is to run in an endless loop. As neither \code{loop} nor \code{while} expressions are supported by the given Ohua implementation, we want/need to add them. 

The frontend language of Ohua is purely functional and hence does not entail while loops. The functional equivalent of a while loop is a recursive function, with the following case distinction: If the condition of the while loop is true, the inner code will be executed and a recursive call is made, otherwise the function returns. After every run of the original loop, local bindings go out of scope and only variables that lived outside the loop 'survive' to the next iteration. In the recursion this is equivalent to using all variables from the outer scope as arguments to the recursive call and return them when the recursion ends. This is illustrated in Figure~\ref{fig:WhileTransform} which shows a simple \code{while} loop in Rust and a functional equivalent using recursion. 

\begin{figure}[H]
\centering
\tabskip=0pt
\valign{#\cr
    \hbox{%
    \begin{subfigure}{.45\textwidth}
    \centering
     \begin{minted}[fontsize=\small]{rust}
fn algo() {
  let mut i:i32 = 0;
  let mut stateObj:State = State::new();
  while check(i) {
    let local = use(i);
    stateObj.mutate_with(local);
    i = i + 1
  }
  stateObj
}
     \end{minted}
    \end{subfigure}%
  }
  \cr
  \noalign{\hfill}
    \hbox{%
    \begin{subfigure}{.45\textwidth}
    \centering
    \begin{minted}[fontsize=\small]{rust}

fn algo() {
  let mut i:i32 = 0;
  let mut stateObj:State = State::new();
  (i , stateObj) = rec_while(i, stateObj)
  stateObj
}

fn rec_while(i:i32, state:State) 
  -> (i32, State) {
  if check(i){
    let local = use(i);
    state.mutate_with(local);
    i = i + 1;
    rec_while(i, state)
  } else {
    (i, state)
  }
}

    \end{minted}
    \end{subfigure}%
  }
  \vfill
  \cr
}
\caption{To express a While-Loop in the purely functional Ohua frontend language, we can rewrite it to a recursive function}
\label{fig:WhileTransform}
\end{figure}

How to:
\begin{itemize}
    \item We note that 
    \begin{enumerate}
        \item want that transformation to be valid/suitable for all language integrations
        \item If we did it at the specific language level, we would have to introduce a new, recursive function, which is complicated from inside processing a function. However one of the first level of transformations inside Ohua is inlinening algorithms as \code{let} definitions  
    \end{enumerate}
    \item So we augment Ohuas Frontend Language with a \code{While condition body} expression, where \code{condition} and \code{body} are the respective components of the original while-loop transformed to expressions of the frontend language.
\end{itemize}

\section{The M3x Backend}
\begin{itemize}
    \item we use one tile for the prototype
    \item we use 'gates' for communication
    \item \todo[inline]{Exact description of gates (e.g. capacities, how do they match our expected data format). If we tried to realize the communication between application and interface as one structure containing a) package payloads and b) commands we might run into capacity problems with the gates}
    \item \todo[inline]{How do we realize credit passing. (How) does it influence performance}
\end{itemize}

\textbf{Important Note}: We do NOT compile from one operating system to another. M$^3$ already runs a slightly adapted version of smoltcp as its network stack. Further it supports all necessary system interfaces and libraries smoltcp requires, in particular M$^3$ offers the required libc calls to connect the application to system sockets and on the other hand smoltcp does not require the rust standard library, which M$^3$ currently does not support. 

In general though different OSes provide different abstractions for OS services. The compilation, as is, takes care of this as far as it concerns i) the abstraction of processes, their creation and communication and ii) the control flow and shared memory usage among/of compiled components. However, this does not account for all necessary adaptations in cross compiling because it is restricted to the compile scope and some specific interactions with the OS. \\
\bigskip

What does this mean?:\\
Whenever the program to be compiled accesses OS services e.g. allocates memory, binds a system level socket etc., those accesses might become invalid for the compiled application running in the target OS.\\
\bigskip

What to do?:\\
If we want to account for this in the future we need to:
\begin{enumerate}
    \item extend the backend transformation to scan for relevant function calls and replace them
    \item do this for all components of the compiled software i.e. not only inside the compile scope but actually for all library files and dependencies that are not already part of the target OS infrastructure
    .  
    \item \todo[inline]{There might be dragons}
    \item FlexOS \todo[inline]{Check Cubicle} handle this problem by providing the programmer a) with a fixed set of supported libraries they can use and b) annotations for system calls, such as memory allocation the compiler can use to automatically replace system call approppriately for the target system
    \item \means most likely supporting this will rely on information from the programmer (e.g. also, when we compile something for the cloud and for example memory allocation or file opening in a task must be adapted to the respective cloud API)
\end{enumerate}


\section{Learnings/Abstract description}
In this section, we consider the concrete changes to smoltcp in the abstract. We try to derive general patterns from the concrete changes how we can transform general imperative code in such a way that stateful objects become composable nodes of a dataflow graph. 

In the case of smoltcp, the previously defined goal was to create a distributed application from three components, the 'server application' itself, the network stack, and the device abstraction. In addition, it was known which parts of smoltcp belonged to which of these components.

\todo[inline]{How to get to the starting states}
\begin{itemize}
    \item we need initial knowledge i.e. compiler doesn't know/ can not infer
    \item at the lowest possible level programmer would just say sockets + interface = component
    \item in that case the transformations include
    \begin{itemize}
        \item form a common object (struct in this case) with fields for each inner component \rust{struct NWComponent {interface: Interface, sockets: Sockets}}, this is already tricky as we need to know how to form this and how many of each subcomponent would be there and how to address them
        \item accessing the inner components directly must not be possible any more \means generate wrapper functions for each call to inner components \means \rust{interface.poll()} becomes \rust{nw_component.poll()} where the later calls the former as  \rust{self.interface.poll()}
        \item \question{How to derive "addressing" when there's more than one of each subcomponent}
    \end{itemize}
\end{itemize}


For the following considerations, we assume that S1, S2 and S3 are stateful objects in the input language that should form the components of the target program. We now look at different possible control flows in which these components occur and describe which transformations are necessary to compile the resulting control flow to a data flow graph with local states. 

Patterns to describe:
\begin{enumerate}
    \item\label{transf:mergeAdj} Composable State Nodes have exactly one 'entry' and 'exit', so \rust{let a: i32 = S1.do(x:i32); let b: i32 = S1.make(y:String)} 
    becomes \rust{let (a, b): (i32, i32) = S1.do_make(x:i32, y:String);} with \rust{fn do_make(&mut self, x:i32, y:String)-> (i32, i32) {let a = self.do(x); let b = self.make(y); (a,b)}}. This is sensitive to order i.e. \rust{make_do()} is another function. This can easily blow up since were talking about ordered powersets of multisets e.g. \rust{make_do_do_make()}
    \item\label{transf:pullUp} States or references to States must not be arguments to functions or methods outside the compile scope. For pure functions this means inline code in the compile scope and proceed. Actually for stateful calls \rust{let a = S1.do(S2); } it means the same in the first step, thereby replacing all \rust{self} references by \rust{S1}, in a second step we actually return to step~\ref{transf:mergeAdj}. This is how \rust{dispatch_before} and \rust{dispatch_after} originated in principle.
    \item state is used two times in a loop \means merge the functions as described in Step~\ref{transf:mergeAdj}
    \item\label{transf:remRefs} We can't pass references in a distributed scenario \means \rust{let pat = fun(&S1, args)} becomes  \rust{let (pat, S1) = fun(S1, args)}, all arguments have to be `moved` in rust parlance and rebound at function return. 
    \todo[inline]{This doesn't work in loops. Why? Describe \rust{Option<S1>} workaround.}
    \item States are used alternately: If we have \rust{S1.act(); S2.do_something(); S1.again()} we needed to a) conserve the state \rust{S1} while \rust{S2} is called, which is not how DFGs work and b) have two nodes for \rust{S1} doing something and send \rust{S1} around which contradicts the state-or-data separation/state locality. So what we need to, as shown in egress and ingress, do a recursion on the 'outer' state call, i.e. first of all we transform to \rust{if entry {S1.act(); S2.do_something(); recur(S1, S2, entry=false)} else { S1.again()}}. Obviously we still have to merge the two functions \rust{S1.act()} and \rust{S1.again()} but this time the resulting arguments and return values must be sum, not product types of the merged function. 
    \todo[inline]{ALL of this sounds like completely solved category theory composition}
    \item Closures and recursion: It would be nice if we could implement the transformation described above as closures because this saves us the problem of identifying required Trait bounds and lifetime. However, recursive closures are tricky, because closures are stack allocated and stack+recursion is not a good combination. So we need to use functions. 
    \question{Can we generally derive trait bounds and lifetimes for those functions?}
    \note{I think so, at least it worked for poll}
    \question{ Can we circumvent using inner functions?}
    \note{Nope. inner functions need trait bounds and lifetimes as well}
    
\end{enumerate}




\chapter{Related Work}
\label{Chapter:Related}
\section{Flexible OS approaches}
\label{sec:FlexibleOSes}
The problem of overcoming disadvantages and potentially combining advantages of microkernel-based operating systems and unikernels has been addressed before. For unikernels disadvantages were a lack of isolation among components and the necessity to adapt to the component the library OS provides. Microkernels on the other hand provide strong isolation but only at the cost of significant overhead for IPC calls and again the necessity to adapt the existing code not to provided components but to communication primitives provided by the kernel. To avoid unnecessary overhead, programmers may also need to refactor the general structure of their code to minimize inter process communication. \\

One example approach is CubicleOS~\cite{sartakov2021cubicleos}. It provides three main, new abstractions to overcome the problems of the isolation vs. overhead trade-off. Those abstractions are 
\begin{enumerate}
    \item \emph{cublicles} used to define memory-isolated processes (components)
    \item \emph{windows} used to define temporary memory sharing among trusted components, and
    \item \emph{cross-cubicle calls} used to implement control flow integrity among \emph{cubicles}
\end{enumerate}
Memory isolation is implemented using Intel's Memory Protection Keys (MPK)\cite{intel64and}. A mechanism implemented in the ISA that manages access rights to virtual page tables based on keys assigned to processes. The current implementation of CubicleOS is based on the Unikraft library OS\cite{kuenzer2021unikraft}. The programmer has to specify the components that should become \emph{cubicles}. During the build process, function call among \emph{cubicles} are identified. To enforce isolation the build system of CubicleOS generates enveloping functions for those calls. These enveloping functions 
called \emph{cross-cubicle calls} implement the context switch between \emph{cubicles} at runtime. Applications using CubicleOS can run on standard Linux. To enforce memory isolation, CubicleOS comes with two runtime components one loading the components with according memory rights, one managing memory access rights. \\

The main adaptations the programmer is required to make are a) using Unikraft components b) defining the \emph{cubicles} and c) defining exceptions from the memory isolation. Exceptions are needed to improve performance and lower the overhead of context switches, when isolation is not desired. Those exception cases are either whole \emph{cubicles} or just data structures shared for particular calls among components. \emph{Cubicles} that are used frequently and are trusted can be declared 'shared' meaning  that there will be no context switches upon calls to any of their functions or usage of static constants. Data structures can be shared using CubicleOSs API for \emph{window}s, which enables the programmer to specify memory locations and sizes and the coded sections for which they should be shared. An example of this API is shown in Figure~\ref{fig:CubicleAPI}, adapted from the original publication.

\begin{figure}[H]
    %\centering
    \begin{subfigure}[b]{0.45\textwidth}
         \includegraphics[width=\textwidth]{figures/cubicle_example_original.png}
         \caption{Original code of application FOO calling library BAR}
         \label{cubicleOriginal}
     \end{subfigure}
     \hfill
     \begin{subfigure}[b]{0.45\textwidth}
         %\centering
         \includegraphics[width=\textwidth]{figures/cubicle_example_windows.png}
         \caption{Annotated code using Cubicles \textit{windows} to share memory after compilation}
         \label{cubicleWindow}
     \end{subfigure}
    \caption{To use Cubicle the developer basically needs to surround calls to other components, in this case BAR with \textit{windows}. Cubicle will derive isolated components and their (legitimate) interaction}
    \label{fig:CubicleAPI}
    \end{figure}

The aim of FlexOS~\cite{lefeuvre2021flexos} is to provide an easy way to exchange isolation primitives for existing code without (extensive) rewrites. Like CubicleOS it is based on an adapted version of the Unikraft library system. The two main primitives the FlexOS API provides are \emph{abstract gates} and \emph{abstract shared data}. In order to compile code with FlexOS, the programmer must replace all function calls between components with \emph{abstract gate} definitions and all shared memory areas with \emph{abstract shared data} definitions. These adjustments have also been made in the adapted system libraries. In addition the FlexOS build system needs a configuration file, which defines among other things the isolation mechanism and the division of the components. During compilation, the abstract definitions are replaced by concrete mechanisms of the target memory isolation techniques, which are MPK, Software Guard Extensions (Intel SGX) and Extended Page Tables \cite{intel64and}. \\

In comparison to our approach, FlexOS and CubileOS solve a very similar problem with very similar constraints for the programmer. She has to define components and their communication explicitly and the adaptation to a concrete architecture is only possible through compilation, because she has to select from suitable system calls and libraries in the input already. In contrary to those systems, we do not provide an API to explicitly allow data sharing via references. It is possible to use reference sharing in Ohuas input programs, as only explicit reference passing is excluded. However Ohua will tread arguments passed by reference the same way as arguments passed by value. So if the resulting program is functional and if pass-by-reference is any more efficient than pass-by-value depends on the target architecture. Which also means we could not effectively use FlexOS or CubileOS as backend integrations.


\section{Compiling to State Local Programs}

Of course, many of the problems and approaches that emerged in the implementation are not new. Denationalization as a concept for mapping higher-order functions to serializable data types was first presented in the work of Reynolds~\cite{reynolds1972definitional}. Building on that concept, as well as Danvy~\cite{danvy2008defunctionalized} discussed how defunctionalization and refactoring to continuation passing style (CPS) can be used to transform programs with structural operational semantics (including interruptions and errors) to reduction semantics. The author integrates their results with previous work yielding the transformation based equivalence graph shown in Figure~\ref{fig:transformationsDanvy}. Given those results we can explore and maybe better describe in future what our transformations should be, for example it would also be conceivable to defining stateful objects as abstract machines within the DFG. In a subsequent work Danvy and Millikin~\cite{DANVY2009534} present \emph{Refunctionalization} as the left inverse of Defunctionalization. This transformation can be applied as an intermediate step, if programs are not directly amenable to defunctionalization. As the authors describe it ``A program can fail to be in the image of defunctionalization if there are multiple points of consumption for elements of a data type or if the single point of consumption is not in a separate function´´. Considering the initial structure of the \rust{poll} function, the transformation described by Danvy and Millikin might also be necessary to formalize the findings in this work.

\begin{figure}[H]
\centering
   \includegraphics[width=0.6\textwidth]{figures/transformations_danvy.png}
\caption{Graph representing the inter-derivability relation of different abstract models for computation, taken from Danvy~\cite{danvy2008defunctionalized}}
\label{fig:transformationsDanvy}
\end{figure}

The paper \emph{Automatically Restructuring Programs for the Web}~\cite{graunke2001automatically} also applies refactoring to CPS followed by defunctionalization. They target the transformation of local, interactive programs to web CGI programs. As with the \stack{.poll} function in our case, the control flow in the source program is continuous in or above the stack frame of the server functions, while the target setting requires explicit handling of both the control flow and the stack state upon leaving an reentering the server function. The paper presents a 
 prototypical implemented automated transformations, to turn a local interactive program into a CGI program. Interestingly in Scheme, the implementation language used, continuations are first-class members and could be stored and reapplied to stack directly assigning each a specific URL, such that the client request can directly address the continuation. Problem was that this was a) language specific and b) had lots of overhead for a distributed garbage collection of stored continuations. So the solution was to embed the control flow into the data flow using defuncionalized representations of entry points. To handle stateful objects, i.e. in general any variable that is reassigned in different invocations of the server, they use the Scheme concept of \code{boxes}. \\
 
 This approach is very similar to what has been done in this work, e.g. by making part of the state of the interface. They must be defined globally, are loaded on each call (from memory or a cookie) and allow all sub-functions of the original server function to access the stateful variable. Considering the server-client setting, the authors also comment on security considerations. The transformations made de facto lead to control flow being integrated into the data flow and, in the case of a web application, can be changed by a malicious user. Even if we aim at a different application scenario, we should evaluate which parts of the control flow actually need to be integrated into the data flow. The security measures proposed by the authors mainly concern the cryptographic protection of the communication and are in this respect possibly a question for further architecture integrations in Ohua. \\

The problem of managing stack based data in event-driven and cooperative programming has  also been addressed in \cite{adya2002cooperative}. Ohua doesn't explicitly aim at event-driven programs. However we have seen that the principle problem of shifting from automatic stack management to a control-flow "re-entering", necessitates to either store and retrieve the execution state explicitly or send parts of it along the data flow, or even passing respectively. The authors discuss necessary refactoring steps to transform single-threaded programs to either a multi-threaded, preemptively scheduled  or even-driven versions. Regardless of the final form, they described the necessity of ``Stack Ripping'', i.e. manually handling the information formerly based on a single function stack. While they use a different vocabulary, the core points of the transformations are actually the same as in the works cited before, facing four main requirements for the  implementation resulting from ``Stack Ripping''
\begin{enumerate}
    \item lifting closures: Parts of functions interleaved with I/O actions or calls to other components need to become language-level functions.
    \item function scope: As more than one function now represent what was one function before the original environment of the code in each sub-function must be manually preserved
    \item automatic variables: variables that where allocated on the stack, but need to survive multiple function calls now, must be allocated on the heap
    \item control structures: Structures like loops or branches lead to additional entry point functions
\end{enumerate}

 Further they describe how in Windows, \code{fibers} and \code{threads} are used to implement interactions between applications using manual or automatic stack management respectively. The identified requirements are very much in line with the necessary refactorings we identified. 


\section{(Automatic) Memory Isolation}
\label{sec:MemoryIsolation}
Besides virtualization and the direct implementation of microkernel-based systems, there are also other approaches to address the security problem of the lack of isolation. One such system is \emph{Hardware-Assisted Kernel Compartmentalization}(HAKC)\cite{mckee2022preventing}. The authors exemplify the problem for Loadable Kernel Modules (LKM), which are themselves not a part of the Linux kernel but loaded and executed in 
kernel space. With many of them providing audio and media processing and drivers, they are a notorious entry point for system compromise exploiting local bugs. In particular they describe an example CVE, where the exploit neither violates memory safety nor control flow integrity and could hence only be prevented by compartmentalization of the modules involved and a defined interfacing to the kernel. HAKC provides an API to define components and a hardware-based runtime system to enforce in-kernel isolation of the defined components. This API allows the programmer to assign compilation units, files or smaller fractions of code into logical units \emph{Cliques} and \emph{Compartments} and define directional memory access and control-flow policies among them. So the actual compartments and access rules are entirely made by the programmer and HAKC itself, does not check or optimize isolation policies. The hardware mechanisms used to enforce the derived policies are called \emph{pointer authentication} (PAC), introduced in ARMv8.3 and \emph{memory tagging extension}(MET), introduced in ARMv8.5-A. The basic principle of TAC is to have pointers cryptographically signed by their legitimate owner, store the signature in the pointer and only allow access based on pointer authentication. MET allows to assign memory regions to different tags and use those groupings to manage access policies. Measured with  different microbenchmarks the overhead for HAKC policy enforcement on the \code{ipv6.ko} LKM imposed a runtime overhead of 1.6\% to 24\% for individual benchmarks.\\

To overcome the difficulty of manually identifying dependencies among components \emph{FlexC}\cite{mckee2022novel} was later presented as an approach to automatically generate the compartmentalization policies that HAKC requires as interface definition among components. The authors argue that in particular in larger system hand-written annotations are error prone and might also lead to inefficient choices. Like HAKC itself, FlexC uses static information from the input code, to generate a data access graph from the input program. To account for indirect data access via pointers the analysis conservatively overestimates potential dependencies. Dynamic data i.e. test runs of the input code can additionally be provided to weighten the data dependencies by frequency of access and size of data. To start the derivation FlexC automatically assigns compilation units (in C usually one ore more files) as smallest entity of isolation. By defining the target number of compartments, the programmer can initialize a greedy fusion of nodes after the initial graph was build, generating a coarser compartmentalization. It is also possible to manipulate the graph using a GUI for FlexC. The output of FlexC is then directly integrated to the compilation process using HAKC. \\

Since most hardware-related, system-level code is still written in C and C++, the approach of \emph{Spons \& Shields}\cite{SponsAndShields} directly builds on these languages. However  they focus rather on a scenario commonly encountered in cloud deployments, namely the usage of \emph{Trusted Execution Environments} based on Intel SGX TEEs. TEEs are increasingly popular for secure cloud deployments and while they where initially used to encapsulate only specific critical user applications, they are now used to move whole OSes into a single secure environment e.g. into an Intel SXG enclave. The identified problem is, again that this creates large code spaces without internal access restrictions and possibly including insecure, third party libraries, which contradicts the defense in depth approach. The next problem the authors notice is that compartmentalization is often enforced along processes. But it is hard to redesign an application to encapsulate different concerns and security levels into different processes. Also the trust model of TEEs does not include the host, while process isolation techniques are based on the primitives the host OS and hardware (MMU) provide. The Spons\& Shields framework (SSF) described in the paper consists of an API to manage the two core abstractions \emph{Spons}, which encapsulate units of execution, e.g. POSIX processes or libraries and \emph{Shields} which define hierarchical memory access regions. The programmer needs to introduce \emph{Spons} and \emph{Shields} according to her requirements in the code and link against the \code{musl} standard C library. During compilation the framework will insert TEE primitives to enable the SSF runtime to enforce memory boundaries between the Spons. To benchmark their technique they used the scenario of a web application consisting of an NGNIX server, a PostreSQL data base (with medical data), an SSL library for crypto and a business logic application in PHP. They compared request latency for a single TEE vs. multi-TEE compartments vs. Shielded deployment and found that shields increase latency by about 1.7 times using Shields while multiple TEEs increased latency about 4.4 times compared to the single TEE version.\\

 \emph{PKRU-Safe}\cite{kirth2022pkru} is an approach with a slightly different target. Instead of vertical compartmentalization it intends to enforce a separation between memory safe and memory unsafe code in Rust applications. The identified problem is that safe languages in this case Rust are already available and it should be possible to interact with unsafe language components without risking memory corruption. So the whole point is to enforce hierarchic/semi-permeable memory isolation between Rust code and unsafe languages. The concept is to have the developer explicitly define the interface between trusted and untrusted parts of the code. This is done via annotations in the projects build file and the untrusted libraries. With per library annotations the adaptation overhead is significantly smaller than in annotating single functions or data access. To analyse the code and track heap allocations PKRU-Safe uses custom plugins to rustc and LLVM tooling. To identify the usage sites of heap allocated data the input program is run with provided profiling input to dynamically identify when the untrusted component accesses data allocated by the trusted component. The gathered access data are used to categorize data allocation into unsafe and safe memory, in particular data that is used by unsafe code is considered belonging to the unsafe code. At runtime this distinction is used to enforce the policy that unsafe code can never access safe memory. To move such data into the untrusted component entirely and to handle the two memory pools at run time the \rust{liballoc} to augment it by dedicated primitives for unsafe memory allocation. The authors describe this runtime behavior as \emph{compartment-aware} heap allocation. By keeping heap section of safe and unsafe memory objects distinct, PKRU-Safe can implement the policies using MPK protection for the trusted heap section. Notably this policy only applies to data on the heap. Stack data is assumed to be protected by other mechanisms. They apply the concept to Servo, a layout-engine written in Rust and tested the performance overhead with different benchmark suites (Dromaeo, Kraken, Octane and JetStream2). They found the altered allocation mechanism to cause an average runtime overhead of 6\% above baseline. The overhead imposed by context switches  involving MPK depended on the concrete benchmark with an average of 11\% and a maximum of about 30\%. \\

Contrary to the Ohua approach, these techniques are currently bound to a particular language and particular hardware mechanisms. On the one hand this enables a deeper analysis of the code, in particular if also dynamic information is considered. It can also lower the imposed overhead, because a distinction between memory sharing and memory transfer can be made. On the other hand it limits the flexibility. All of the approaches require additional programmer information to identify the target components, but also to shape the intended isolation policies. In how far data of one component is still accessible to the other is in case of Ohua, depending on the chosen architecture integration and the programmers compliance to the programming model, in case the architecture permits the usage of shared references. 


\chapter{Evaluation and Lessons learned}
\label{Chapter:Learnings}
\input{chapters/LearningsAndEvalutation}

\chapter{Conclusion}
\label{Chapter:Discussion}
\section{So ... }

Contibutions (so far)
\begin{itemize}
    \item changed type extraction in rust
    \item fixed SSA in Ohua
    \item mechanism to wrap not yet supported code (still in the making)
    \item extend tests in Ohua (now there's tests for variable name scoping, tests for programming model compliance in the making)
    \item \todo[inline]{Fix M3 Backend}
    \item \todo[inline]{maybe fix while}
    \item \todo[inline]{adapt smoltcp refactoring to new version}
    \item \todo[inline]{refactor ingress and poll to compile with Ohua}
    \item \todo[inline]{abstract and describe required code/control flow transformations}
    \item \todo[inline]{maybe sketch an algorithm}
\end{itemize}

\cleardoublestandardpage
\bibliographystyle{plain}
\bibliography{literature.bib}
\cleardoublestandardpage
\listoffigures
\cleardoublestandardpage
\listof{codefigure}{List of Listings}      
\cleardoublestandardpage
\listoftables


\appendix
\chapter{Appendix}
\label{Appendix}
\section{Rust Language Integration}
\label{sec:RustIntegration}

\begin{table}[H]
    \resizebox{\columnwidth}{!}{%
    \begin{tabular}{l c l l}
        \multicolumn{4}{l}{\emph{Block of Statements:}}\\
        $block$ & $::=$ & \textbf{\{}$s;~\ldots ;~s $\textbf{\}} & statements \\
        \multicolumn{4}{l}{\emph{Statements:} }\\
        $s$ & $::=$ & $ e $ & statement returning a value\\
        & $|$ & $ e $\textbf{;} & statement not returning a value\\
        & $|$ & \textbf{let} $pat$ \textbf{=} $e$ & local definition\\ 
        
        \multicolumn{4}{l}{\emph{Expressions:}}\\
        $e$ & $::=$ & $x$ & variable bindings \\
        & $|$ & $\textbf{1},\textbf{2},\textbf{3}, \ldots \ |\ \textbf{true}\ |\ \textbf{false}\ $  & literals \\
        & $|$ & $e$ \textbf{(}$e, \ldots , e$\textbf{)} & function calls \\
        & $|$ & $e$ $callRef$ \textbf{(}$e, \ldots , e$\textbf{)} & method calls \\
        & $|$ & $\textbf{(} e,~\ldots ,~e \textbf{)}\ $ & tuples\\
        & $|$ & $e\ +\ e\ |\ e\ -\ e \ |\ e\ > \ e\ |\ e\ == \ e\ |\ \ldots $ & binary operations \\
        & $|$ & $ -e\ |\ !e\ |\ *e\ $ & unary operations\\
        & $|$ &\textbf{if} $e$ $block$ \textbf{else} $e$& conditional with optional else branch \\
        % & $|$ & \textbf{while} $\ e$  $block$ & \\ % actually while loops are not supported at all
        & $|$ & \textbf{for} $pat$ \textbf{in} $e$ $block$ & \\ 
        & $|$ & \textbf{move} $\textbf{| } arg,~\ldots ,~arg \textbf{ |}\ e$ & closure\\
        &$|$& $block$ & block expression, i.e. block returning a value\\
        \multicolumn{4}{l}{\emph{Call References:}}\\
        $callRef $ & $::=$ & $ z\textbf{.}y\textbf{.}x$ $[gernericArg]$  & namespaced binding  \\
        \multicolumn{4}{l}{\emph{Patterns:}}\\
        $pat$ & $::=$ & $x\ |\ (x,~\ldots ,~x)|\ \_ $ & bindings, tuples or wild cards \\
    \end{tabular}%
    }
    \caption{Subset of Rust grammar, that is accepted by the Rust integration frontend}
    \label{AppTab:FESubset}
\end{table}


\section{Additional Source Code Excerpts}

\begin{figure}[H]
\centering
\begin{minted}[fontsize=\tiny]{rust}
pub fn poll<'a, D>(
    mut ip_stack: Interface<'a>,
    timestamp: Instant,
    mut device: D,
    mut sockets: SocketSet<'static>)
    -> ( Result<bool>, Interface<'a>, D, SocketSet<'a>)
    where D: for<'d> Device<'d>
{

    ip_stack.set_inner_now(timestamp);
    // .. fragment handling

    let mut readiness_has_changed = false;

    loop {
        let processed_any = ip_stack.socket_ingress(device, sockets);
        // Begin of inlined ip_stack.socket_egress()

        let mut emitted_any = false;

        for item  in sockets.items_mut() {
            if ip_stack.item_meta_egress_permitted(item)
            {
                let mut neighbor_addr = None;
                let result:Result<()>;
                let packet_or_ok =
                    // socket function wrapped to happen inside the ip_stack
                    // component
                    ip_stack.match_socket_dispatch_before(item);
                if is_packet(&packet_or_ok) {
                    let (response, response_and_keepalive) =
                        from_packet(packet_or_ok);
                    neighbor_addr = as_optn_addr(&response);
                    let sending_token = device.transmit();
                    if sending_token.is_some() {
                        let local_dispatch_result = ip_stack.inner_dispatch_local(response, None);
                      if let Ok((packet, timest)) = local_dispatch_result{
                            let send_result =
                                device.consume_token(timest, packet, sending_token.unwrap());
                            if send_result.is_ok() {
                                // socket function wrapped to happen inside // the ip_stack component
                                result = 
                                ip_stack.match_socket_dispatch_after(item, response_and_keepalive);
                                emitted_any = true;
                            } else {
                                result = send_result
                            }
                        } else {
                            result = Err(local_dispatch_result.unwrap_err());
                        }
                    } else {
                        net_debug!("failed to transmit IP: {}", Error::Exhausted);
                        result = Err(Error::Exhausted);
                    }
                } else {
                    result = Ok(());
                }

                let maybe_break = ip_stack.handle_result(result, item, neighbor_addr);
                if maybe_break {
                    break
                    }
                }
            }
            // End of inlined ip_stack.socket_egress()

        if processed_any || emitted_any {
            readiness_has_changed = true;
        } else {
            break;
        }
    }
    (Ok(readiness_has_changed), ip_stack, device, sockets)
}
\end{minted}
\caption{After inlining all functions using the \dev{} and re-encapsulating all code that concerns only the \stack{} we get an interleaved usage of \stack{} and \dev{} in the sending loop. Due to multiple mutable borrows of both components in the loop, this code will not compile.}
\label{appFig:egressInlinedOriginal}
\end{figure}

\end{thesisdocument}
\end{document}